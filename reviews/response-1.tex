\documentclass{article}

\usepackage{amsmath}
%\usepackage{amsfonts}
\usepackage{amsthm}
%\usepackage{amssymb}
%\usepackage{mathrsfs}
%\usepackage{fullpage}
%\usepackage{mathptmx}
%\usepackage[varg]{txfonts}
\usepackage{color}
\usepackage[charter]{mathdesign}
\usepackage[pdftex]{graphicx}
%\usepackage{float}
%\usepackage{hyperref}
%\usepackage[modulo, displaymath, mathlines]{lineno}
%\usepackage{setspace}
%\usepackage[titletoc,toc,title]{appendix}
\usepackage{natbib}

%\linenumbers
%\doublespacing

\theoremstyle{definition}
\newtheorem*{defn}{Definition}
\newtheorem*{exm}{Example}

\theoremstyle{plain}
\newtheorem*{thm}{Theorem}
\newtheorem*{lem}{Lemma}
\newtheorem*{prop}{Proposition}
\newtheorem*{cor}{Corollary}

\newcommand{\argmin}{\text{argmin}}
\newcommand{\ud}{\hspace{2pt}\mathrm{d}}
\newcommand{\bs}{\boldsymbol}
\newcommand{\PP}{\mathsf{P}}
\let\divsymb=\div % rename builtin command \div to \divsymb
\renewcommand{\div}[1]{\operatorname{div} #1} % for divergence
\newcommand{\Id}[1]{\operatorname{Id} #1}

\begin{document}

\title{Response to review \#1}
\author{}
\date{}

\maketitle

Thank you Doug for this helpful review!
Original comments are in blue, our responses in black.

\section*{Summary}

\textcolor{blue}{In ‘icepack: a new glacier flow modeling package in Python, version 1.0’, Shapero and co-authors present a promising new ice sheet modeling framework. The
framework contains mechanisms for solving both the prognostic mass balance
equations for updating ice sheet geometry, as well as diagnostic solvers for approximations to the non-linear Stokes’ equations. Throughout both the software
and the manuscript describing it, the authors focus on ensuring usability (readability), a trait that is bound to make this software (and paper) frequently used.
Despite its accessibility, the capabilities of icepack are already impressive, made
all the more so by its explicit design prioritization of easy extensibility.
I find the manuscript to be exceptionally well-crafted, and I think that it
could be published as is. That said, I offer a few suggestions, comments, and
points of clarification below.}


\section*{Minor points}

\begin{description}
\item[L15-18] \textcolor{blue}{It would be nice to have a cited example or three for each of these
suggested use cases. This would help the reader identify the kinds of
practical problems where icepack might fill a need.}

\item[L74] \textcolor{blue}{A low aspect ratio isn’t really an approximation; it’s an existential fact.
The approximation that the first order approximation makes is that vertical resistive stresses (or bridging stresses as they are referred to later
in this manuscript) are small, pressure is hydrostatic, and bed slopes are
small.}

\item[L92] \textcolor{blue}{It would be useful to offer a reference regarding an anisotropic fluidity.}

\item[L100] \textcolor{blue}{Not clear where the Legendre transform enters: the viscous and frictional
dissipation can be read off from Eq. 4.
Eqs. 6 and 12 While I understand that it is convenient to manipulate the
action to reflect the algebraic manipulations to yield the analytical SIA
solution, the break in symmetry between Eq. 6 and Eq. 12 is frustrating,
1given that they both are name ‘gravity’, and that they should in some
sense be the same regardless of which strain rates are assumed to be zero.}

\item[L144] \textcolor{blue}{That the terminal potential term, Eq. 13 disappears is not obvious. This
should probably be clarified, since many readers will be surprised by this.}

\item[L152] \textcolor{blue}{Cite the method of manufactured solutions.}

\item[L159] \textcolor{blue}{The benefit to avoiding complicated 3D meshing should not be under-
stated, in addition to the reduction in the cost of computational solution.}

\item[L188] \textcolor{blue}{This is a bit of a red herring, given that impenetrability is a natural
boundary condition on the incompressibility equation. It can go right
in the action principle, no extra Lagrange multiplier (besides pressure)
needed.}

\item[L367] \textcolor{blue}{I’m surprised by this. For challenging geometries, I’ve always needed to
stabilize even when using implicit Euler. Are you sure that implicit Euler
is unconditionally stable even for non-linear advection like this?}

\item[L453] \textcolor{blue}{Many advances have occurred in the last 5 years regarding gradient de-
scent due to its necessity for optimizing neural networks. These may yet
be useful in this context if you have to deal with a large scale optimization
problem where forming the Hessian becomes prohibitive.}
\item[L459] \textcolor{blue}{Gauss-Newton needs a reference if you’re not going to describe it here.
Eq. 34 It’s worth noting that the Schoof law is phenomenological, and was
selected because it has the right shape and obeys Iken’s bound. As such,
if your sliding law obeys Iken’s bound and looks right, then it’s not any
less valid.}
\item[L580] \textcolor{blue}{I think that having to specify a variational principle for the sliding law
is useful because it guarantees a law that is positive (semi-)definite, as it
must be to be physical.}
\item[L701–713] \textcolor{blue}{I enjoyed reading this paragraph, but I wonder if stabilizing the
Stokes’ equations is the best illustration of the point, given that icepack
does not in fact have Stokes’ equations implemented (although I imagine
it could be done in short order). Maybe stabilizing a transport equation
would be a more appropriate case?}
\end{description}

\end{document}
