\documentclass{article}

\usepackage{amsmath}
%\usepackage{amsfonts}
\usepackage{amsthm}
%\usepackage{amssymb}
%\usepackage{mathrsfs}
%\usepackage{fullpage}
%\usepackage{mathptmx}
%\usepackage[varg]{txfonts}
\usepackage{natbib}
\usepackage{color}
\usepackage[charter]{mathdesign}
\usepackage[pdftex]{graphicx}
%\usepackage{float}
%\usepackage{hyperref}
%\usepackage[modulo, displaymath, mathlines]{lineno}
%\usepackage{setspace}
%\usepackage[titletoc,toc,title]{appendix}

%\linenumbers
%\doublespacing

\theoremstyle{definition}
\newtheorem*{defn}{Definition}
\newtheorem*{exm}{Example}

\theoremstyle{plain}
\newtheorem*{thm}{Theorem}
\newtheorem*{lem}{Lemma}
\newtheorem*{prop}{Proposition}
\newtheorem*{cor}{Corollary}

\newcommand{\argmin}{\text{argmin}}
\newcommand{\ud}{\hspace{2pt}\mathrm{d}}
\newcommand{\bs}{\boldsymbol}
\newcommand{\PP}{\mathsf{P}}

\title{\emph{icepack}: a novel glacier flow modeling package}
\author{Daniel Shapero}
\date{}

\begin{document}

\tableofcontents
\newpage

\maketitle

\begin{abstract}
In this paper we introduce a new software package called \emph{icepack} for modeling the flow of glaciers and ice sheets.
Icepack is built on the finite element modeling library firedrake, which implements the domain-specific language UFL for the specification of weak forms of partial differential equations.
The diagnostic models implemented in icepack use variational formulations that are specified in UFL.
Individual components of each model's action functional can be easily substituted for components of the user's choosing, facilitating experimentation with the model physics.
Additionally, many post-processing and analysis tasks on simulation results amount to the evaluation of some functional.
By using a variational formulation of the model physics, the specification of a problem and the analysis of the solution employ the same conceptual vocabulary.
A third advantage of variational principles is that the action functional itself can be used to define a solver convergence criterion that is independent of the mesh and requires little tuning on the part of the user.
Icepack features a 3D diagnostic model based on terrain-following coordinates and vertical spectral discretization.
This model resolves both plug- and shear-flow components of horizontal ice flow with a minimum of computational expense over 2D, depth-averaged models.
Finally, icepack implements a Gauss-Newton solver for inverse problems that runs substantially faster than the standard BFGS method used in the glaciological literature.
The overall design philosophy of icepack is to enable usability for as wide a swathe of the glaciological community as possible, including both experts and novices in numerical PDE solvers and high-performance computing.
\end{abstract}

\section{Introduction}

Numerical modeling has become an essential part of the workflow of glaciologists across all disciplines.
We highlight four main uses of glacier models in the literature:
\begin{enumerate}
    \item predicting future glacier extent and estimating the sea-level rise contribution from glacier dynamics,
    \item exploring poorly-understood aspects of glacier physics, such as hydrology and calving,
    \item estimating unobservable quantities, such as bed friction or rheology, from observational data, and
    \item reconstructing what glaciers of the near- or distant-past may have looked like.
\end{enumerate}
Nearly all glaciologists, from graduate students to senior researchers, need to use numerical models at some point in their career.
Several glacier flow models already exist and are effective in the hands of experts.
These models are usually written in compiled programming languages such as C, C++, and Fortran, which afford an unsurpassed degree of computational speed.
Many researchers in glaciology, however, receive little or no formal programming training, much less in these languages, and are instead self-taught in either Python or MATLAB.
The ubiquity of C, C++, and Fortran in scientific computing can create a barrier to entry for using numerical models to glaciologists who are not experts in modeling as such.
We wanted to make a tool that would be accessible also to non-experts.

\textcolor{red}{We focus in this paper only on novel aspects of icepack that differ substantially from existing packages.}

The glacier flow modeling package closest in spirit to icepack is VarGlaS \citep{brinkerhoff2013data}.
VarGlaS is implemented using the finite element modeling package FEniCS \citep{logg2012automated}.
The Firedrake project began as an outgrowth of FEniCS and both packages implement the same domain-specific language for specifying weak forms of PDE.
Icepack improves upon the groundwork laid in VarGlaS through its use of tensor product finite elements on extruded meshes, both of which are only available in firedrake \citep{bercea2016structure, mcrae2016automated}.
These two features enable the definition of a much simpler 3D flow model that we will describe in the next section.

The two main components of a glacier flow model are a \emph{diagnostic} and a \emph{prognostic} equation.
The diagnostic equation prescribes the ice velocity through a time-independent, nonlinear, elliptic partial differential equation.
The inputs to the diagnostic equation are the ice thickness, surface elevation, rheology, ice velocity at the inflow boundary, and the coefficient of friction for contact with the bed and side walls.
The output of the diagnostic equation is the ice velocity throughout the entire region of interest.
The prognostic equation prescribes how the ice thickness evolves through conservation of mass.
The inputs to the prognostic equation are the current value of the ice thickness, the ice velocity, and the surface and basal mass balance.
The output is the ice thickness at a later time.
Mathematically, these two coupled PDEs can be thought of as a differential-algebraic equation.

The rheology and friction coefficient may also be described in terms of other fields.
For example, the rheology can be parameterized in terms of the ice temperature, englacial water content, and damage state.
Likewise, the friction coefficient can be described in terms of a geological roughness factor and the subglacial water pressure.
The diagnostic and prognostic equations can then be supplemented with evolution equations for these fields, for example the heat equation for temperature, or a hydrology model for subglacial water pressure.


\section{Diagnostic models}

There are two diagnostic models implemented in icepack.
The \emph{shallow stream approximation} (SSA) is a 2D model describing the depth-averaged velocity of a fast-flowing grounded ice stream or floating ice shelf \citep{macayeal1989large}.
The SSA model is appropriate where the sliding velocity is close to the surface velocity, or in other words where the ice is nearly in plug flow.
Plug flow is a good approximation in fast-flowing ice streams and outlet glaciers near the margins of an ice sheet, but deep in the interior the flow is mostly by vertical shear.
The \emph{first-order} or \emph{Blatter-Pattyn} (BP) approximation, by contrast, is a 3D model describing the horizontal velocity \citep{blatter1995velocity, pattyn2003new}.
The only approximation in the BP model is that the flow has a low aspect ratio -- the thickness of the glacier is much less than its horizontal extent.
This approximation may be questionable around, say, the main trunk of Jakobshavn Isbrae in Greenland, which flows through a very deep and narrow trough.
Even Jakobshavn has an aspect ratio on the order of 1/5 and almost all glacier flows have an aspect ratio less than 1/10 or even 1/20.

Both the SSA and BP models in icepack are described through \emph{action principles} \citep{dukowicz2010consistent}.
Rather than describe the velocity as the weak solution of a nonlinear PDE, an action principle instead states that the velocity minimizes a certain nonlinear functional, called the action.
The action consists of four terms:
\begin{align}
    & \text{action} = \iint\text{stress} \times \text{strain rate}\ud z\ud x - \int\text{basal friction} \times \text{sliding velocity}\ud x \nonumber \\
    & \quad - \iint\text{surface slope}\times\text{velocity}\ud z\ud x - \iint\text{ocean pressure}\times\text{velocity}\ud z\ud \gamma
    \label{action-functional}
\end{align}
where $\ud z\ud x$ denotes integration over the entire glacier, $\ud x$ denotes integration over the glacier footprint, and $\ud z\ud\gamma$ over the side wall boundary.
The action has units of power (energy/time) and can be related to the rate of decrease of the thermodynamic free energy.
Moreover, the energy lost to viscous and frictional heating can be calculated from the action and its Legendre transform.
The action principle can be viewed as a consequence of the Onsager reciprocity relations for systems near to equilibrium \citep{de2013non}.

Irrespective of any deep connections with thermodynamics, the existence of an action principle is a tremendous convenience for numerical analysis.
For viscous flow problems near to steady-state, the action is \emph{convex} as a function of the ice velocity, i.e. its second derivative is positive-definite.
Convexity implies that the action functional has a unique minimizer and that, with an appropriate line search strategy, Newton's method will converge from any initial guess.
Minimizing a convex action functional is vastly more convenient numerically than solving a general nonlinear equation, although the two formulations are equivalent.


\subsection{Implementation with firedrake}

\subsection{Substituting model components}

Many aspects of glacier physics are not completely understood.
For example, the most common description of basal friction assumes that glacier sliding occurs through regelation, in which case the basal shear stress can be written
\begin{equation}
    \tau_b = -C|u|^{\frac{1}{m} - 1}u
\end{equation}
with $m = 3$ \citep{weertman1957sliding}.
Many authors have argued that, in fast-flowing regions of the ice sheet, glacier sliding occurs instead by plastic deformation within the subglacial sediments \citep{tulaczyk2000basal}.
For plastic sliding, the shear stress is dependent on the yield strength of the subglacial sediments and not on the sliding speed, in which case $m = \infty$ would be most appropriate.
Finally, the Schoof sliding law aims to reconcile the two modes of sliding \citep{schoof2005effect}.
Below a critical sliding speed, the shear stress is aymptotic to $|u|^{1/m}$ as in the Weertman sliding law, but above the critical speed the shear stress is independent of the sliding speed.

The Weertman and plastic sliding laws possess the same functional form but differ only in the value of a single scalar parameter $m$.
The Schoof sliding law, on the other hand, has a totally different functional dependence on the velocity.
Several authors, including Schoof, have proposed that the basal shear stress is also a function of the effective pressure $N = \rho gh - p_w$ within the subglacial hydrological system \citep{budd1979empirical, schoof2005effect}.
Implementing these more sophisticated mathematical models would require adding an extra argument to the procedure for solving the diagnostic equation.

One of our goals with icepack is to facilitate experimentation with the model physics.
To support use cases like implementing the Schoof sliding law, it must be possible to completely alter the functional form of a given model physics component.
For uses cases like explicitly adding the dependence of basal shear stress on hydrology, it must also be possible to add entirely new fields to a given model physics component.
In programming terms this amounts to changing the number of arguments to the function that calculates basal shear stress.
For a library developed in C or Fortran, the user would then also have to change the signature of the diagnostic solve function.
In C++ one could obviate the need to modify the diagnostic solve routine by using variadic templates.
But variadic templates are a more advanced language feature and using them would arguably present a steep learning curve for novices.

In icepack, any given model physics component can be substituted for a parameterization of the user's choosing.
Every flow model in icepack is represented by a Python object, and each model object is initialized with a set of functions to calculate the various terms of the action functional.
For example, the ice stream model is initialized with functions to calculate the contributions due to viscosity, basal friction, side wall drag, and gravity.
Sensible parameterizations are selected by default.
If the user does wish to change a given component, they can pass a function that they write to the initializer for the model object.
When the diagnostic solve procedure is invoked, all of its arguments are then passed on by keyword to the method that calculates the action functional.
Any fields that are unnecessary for calculating a given component are simply ignored.
For example, the rheology and friction coefficient are passed to the function that calculates the gravitational driving power, and this function simply makes no use of these arguments.
The functional form can be changed entirely and new arguments can be added, with the only restriction that they are passed as keyword and not as positional arguments.

This approach to making the physics models more extensible restricts us to using keyword arguments instead of positional arguments.
In this case, passing arguments by keyword enhances readability and comprehensibility.
The user only needs to know the argument names, which are chosen to agree with what symbols are commonly used for each ($C$ for friction, $A$ for rheology, $N$ for effective pressure, etc.).
The order of the arguments is arbitrary and immaterial.
For defining, say, the bilinear form that represents a non-symmetric PDE, the order of the arguments does have some inherent mathematical significance.


\subsection{Convergence criteria}

Calculating the ice velocity from the geometry and other input fields involves minimizing a convex functional.
The minimizer can only be approximated, so we employ iterative procedures based on Newton's method with a line search.
We must then choose some metric to decide when the current guess for the velocity is good enough to stop the iteration.

A convergence criterion that works equally well independent of the mesh, finite element discretization, and the quality of the initial guess can be defined based on the idea of the \emph{Newton decrement} \citep{nocedal2006numerical}.
The search direction $v_k$ at step $k$ for Newton's method is defined as
\begin{equation}
    v_k = -\ud^2J(u_k)^{-1}\ud J(u_k).
\end{equation}
Since the second derivative operator $\ud^2J(u_k)$ is positive-definite, this is a descent direction for the action $J$:
\begin{equation}
    \ud J(u_k)\cdot v_k < 0.
\end{equation}
The absolute value of the quantity in the last equation is defined as the Newton decrement.
For $u_k$ sufficiently close to the true solution $u$, the Newton decrement roughly tells us how much we can expect the action to decrease:
\begin{equation}
    J(u_k) - J(u) \approx \frac{1}{2}|\ud J(u_k)\cdot v_k|.
\end{equation}
We can then use the Newton decrement to decide when to stop the iteration.

As shown in equation \eqref{action-functional}, the action has units of power and is the sum of the dissipation due to viscosity, friction, gravitational driving, and restrain by ocean pressure at the terminus.
The viscous and frictional terms are convex and positive functions of the velocity, while the gravitational and terminus stress terms are linear in the velocity and can be of either sign.
If we define the \emph{scale functional}
\begin{equation}
    K(u) = \text{viscous dissipation} + \text{frictional dissipation}
\end{equation}
as only the positive parts of the action, then the convergence criterion
\begin{equation}
    |\ud J(u_k)\cdot v_k| < \epsilon K(u_k)
\end{equation}
is independent of the discretization.
The intuition behind this criterion is that iteration is halted when the expected decrease in the action functional is much smaller than the positive part of the action itself.

We have found empirically that, with this criterion and the Newton solver implementation in icepack, the iteration usually converges to machine precision in around 8 steps.
For exceptionally bad initial guesses the iteration count can reach as high as 20 but rarely more.
We also observe the expected doubling of the number of accurate digits in the value of the action once the velocity guesses are within the convergence basin of the true solution.
Other convergence criteria, such as using relative change in the velocity guesses, can terminate prematurely when the initial guess is very far outside the quadratic convergence basin.

The numerical solvers in icepack have been designed so that users who are not familiar with numerical optimization need not be confronted with a possibly bewildering array of algorithmic parameters.
Consequently, sensible defaults have been chosen for the Armijo and Wolfe criteria \citep{nocedal2006numerical}, and the tolerance for the line search is chosen based on that of the outer-level Newton iteration.
The Newton search direction is calculated using a direct factorization solver rather than, say, the conjugate gradient algorithm, as the use of another iterative method would introduce yet another algorithmic parameter.
The existing algorithmic parameters can nonetheless be changed from their defaults by passing keyword arguments for advanced users who are interested in performance optimization.


\section{Prognostic model}


\section{Data assimilation}


\section{Demonstrations}


\section{Testing}


\section{Usability}


\section{Discussion}

\bibliographystyle{plainnat}
\bibliography{icepack.bib}

\end{document}
