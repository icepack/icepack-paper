%% Copernicus Publications Manuscript Preparation Template for LaTeX Submissions
%% ---------------------------------
%% This template should be used for copernicus.cls
%% The class file and some style files are bundled in the Copernicus Latex Package, which can be downloaded from the different journal webpages.
%% For further assistance please contact Copernicus Publications at: production@copernicus.org
%% https://publications.copernicus.org/for_authors/manuscript_preparation.html


%% Please use the following documentclass and journal abbreviations for preprints and final revised papers.

%% 2-column papers and preprints
\documentclass[journal abbreviation, manuscript]{copernicus}



%% Journal abbreviations (please use the same for preprints and final revised papers)


% Advances in Geosciences (adgeo)
% Advances in Radio Science (ars)
% Advances in Science and Research (asr)
% Advances in Statistical Climatology, Meteorology and Oceanography (ascmo)
% Annales Geophysicae (angeo)
% Archives Animal Breeding (aab)
% ASTRA Proceedings (ap)
% Atmospheric Chemistry and Physics (acp)
% Atmospheric Measurement Techniques (amt)
% Biogeosciences (bg)
% Climate of the Past (cp)
% DEUQUA Special Publications (deuquasp)
% Drinking Water Engineering and Science (dwes)
% Earth Surface Dynamics (esurf)
% Earth System Dynamics (esd)
% Earth System Science Data (essd)
% E&G Quaternary Science Journal (egqsj)
% European Journal of Mineralogy (ejm)
% Fossil Record (fr)
% Geochronology (gchron)
% Geographica Helvetica (gh)
% Geoscience Communication (gc)
% Geoscientific Instrumentation, Methods and Data Systems (gi)
% Geoscientific Model Development (gmd)
% History of Geo- and Space Sciences (hgss)
% Hydrology and Earth System Sciences (hess)
% Journal of Bone and Joint Infection (jbji)
% Journal of Micropalaeontology (jm)
% Journal of Sensors and Sensor Systems (jsss)
% Magnetic Resonance (mr)
% Mechanical Sciences (ms)
% Natural Hazards and Earth System Sciences (nhess)
% Nonlinear Processes in Geophysics (npg)
% Ocean Science (os)
% Primate Biology (pb)
% Proceedings of the International Association of Hydrological Sciences (piahs)
% Scientific Drilling (sd)
% SOIL (soil)
% Solid Earth (se)
% The Cryosphere (tc)
% Weather and Climate Dynamics (wcd)
% Web Ecology (we)
% Wind Energy Science (wes)


%% \usepackage commands included in the copernicus.cls:
%\usepackage[german, english]{babel}
%\usepackage{tabularx}
%\usepackage{cancel}
%\usepackage{multirow}
%\usepackage{supertabular}
%\usepackage{algorithmic}
%\usepackage{algorithm}
%\usepackage{amsthm}
%\usepackage{float}
%\usepackage{subfig}
%\usepackage{rotating}


\begin{document}

\title{\emph{icepack}: a new glacier flow modeling package in Python, version 1.0}


% \Author[affil]{given_name}{surname}

\Author[1]{Daniel}{Shapero}
\Author[2]{Jessica}{Badgeley}
\Author[2]{Andrew}{Hoffmann}
\Author[1]{Ian}{Joughin}

\affil[1]{Polar Science Center, Applied Physics Laboratory, University of Washington, Seattle, WA, USA}
\affil[2]{Department of Earth and Space Sciences, University of Washington, Seattle, WA, USA}

%% The [] brackets identify the author with the corresponding affiliation. 1, 2, 3, etc. should be inserted.

%% If an author is deceased, please mark the respective author name(s) with a dagger, e.g. "\Author[2,$\dag$]{Anton}{Aman}", and add a further "\affil[$\dag$]{deceased, 1 July 2019}".

%% If authors contributed equally, please mark the respective author names with an asterisk, e.g. "\Author[2,*]{Anton}{Aman}" and "\Author[3,*]{Bradley}{Bman}" and add a further affiliation: "\affil[*]{These authors contributed equally to this work.}".


\correspondence{Daniel Shapero (shapero@uw.edu)}

\runningtitle{icepack}

\runningauthor{Shapero, Badgeley, Hoffmann, Joughin}





\received{}
\pubdiscuss{} %% only important for two-stage journals
\revised{}
\accepted{}
\published{}

%% These dates will be inserted by Copernicus Publications during the typesetting process.


\firstpage{1}

\maketitle



\begin{abstract}
We introduce a new software package called \emph{icepack} for modeling the flow of glaciers and ice sheets.
Icepack is built on the finite element modeling library Firedrake, which uses the domain-specific language UFL for describing weak forms of partial differential equations.
The diagnostic models in icepack are formulated through action principles that are specified in UFL.
The components of each action functional can be substituted for different forms of the user's choosing, which makes it easy to experiment with the model physics.
The action functional itself can be used to define a solver convergence criterion that is independent of the mesh and requires little tuning on the part of the user.
Icepack includes the 2D shallow ice and shallow stream models.
We have also defined a 3D hybrid model based on spectral semi-discretization of the Blatter-Pattyn equations.
Finally, icepack includes a Gauss-Newton solver for inverse problems that runs substantially faster than the BFGS method often used in the glaciological literature.
The overall design philosophy of icepack is to be as usable as possible for a wide a swath of the glaciological community, including both experts and novices in computational science.
\end{abstract}


% \copyrightstatement{TEXT} %% This section is optional and can be used for copyright transfers.


\introduction  %% \introduction[modified heading if necessary]

Nearly all glaciologists, from graduate students to senior researchers, need to use numerical models at some point in their career.
Several software packages for glacier flow modeling already exist and are effective in the hands of experts.
We highlight four main uses of glacier models in the literature:
\begin{enumerate}
    \item projecting future glacier extent and estimating the sea-level rise contribution from glacier dynamics,
    \item exploring aspects of glacier physics, such as hydrology and calving, that are not completely understood,
    \item estimating unobservable quantities, such as bed friction or rheology, from observational data, and
    \item reconstructing what glaciers of the near- or distant-past may have looked like.
\end{enumerate}
To accomplish these tasks, modeling tools are usually written in compiled programming languages such as C, C++, and Fortran for reasons of computational efficiency.
Many glaciologists receive little or no formal programming training, much less in these languages, and are instead self-taught in either Python or MATLAB.
The ubiquity of C, C++, and Fortran in scientific computing can create a barrier to entry for glaciologists who are not experts in high-performance computing.
In this paper, we introduce a new Python software package for glacier flow modeling called \emph{icepack}.
Our goal is to make a tool that will be both accessible to novices and more productive for experts.

The glacier flow modeling package closest in spirit to icepack is VarGlaS \citep{brinkerhoff2013data}.
VarGlaS is implemented using the finite element modeling (FEM) package FEniCS \citep{logg2012automated}, whereas icepack is built on top of Firedrake, which began as an outgrowth of FEniCS.
Both packages share a similar goal of saving users from manually writing low-level code for assembling the systems of equations that discretize their model physics.
Instead, users describe the weak form of the partial differential equations they wish to solve using a high-level domain-specific language (DSL) called the \emph{Unified Form Language} or UFL \citep{alnaes2014unified}.
This DSL is embedded entirely into Python, i.e. the complete syntax of UFL can be mapped directly onto overloaded operators in the Python programming language.
Both FEniCS and Firedrake then generate optimized C or C++ code to assemble the discretized system of equations from this symbolic description of the problem \citep{kirby2006compiler, rathgeber2016firedrake}.
Combining a DSL and a code generator frees users from the very error-prone process of writing these assembly kernels themselves, and makes the syntax of the code align more closely to the syntax of written mathematical expressions.
Additionally, having a high-level symbolic description of the problem makes it possible to automatically derive tangent linear models and adjoints \citep{mitusch2019dolfin}.

Icepack improves upon the groundwork laid in VarGlaS in three main respects.
First, icepack includes a simple 3D flow model that uses several features only available in Firedrake: extruded meshes and tensor product finite elements \citep{bercea2016structure, mcrae2016automated} (see \S\ref{sec:physics-hybrid-model}).
Second, icepack's architecture is designed to make it easy for users to alter the various physics components, such as the rheology and basal friction, of any of its constituent models (see \S\ref{sec:physics-substitution}).
Finally, the inverse solver in icepack uses the Gauss-Newton method, which converges faster and more reliably than the BFGS method used in VarGlaS (see section \S\ref{sec:numerics-inverse-solvers}).


\section{Physics}

The two main components of a glacier flow model are a \emph{diagnostic} and a \emph{prognostic} equation.
The diagnostic equation prescribes the ice velocity through a time-independent, nonlinear, elliptic partial differential equation (PDE).
The prognostic equation prescribes how the ice thickness evolves through conservation of mass.
Mathematically, these two coupled PDEs can be thought of as a differential-algebraic equation.

The rheology and friction coefficient are functions of other fields that have their own evolution equations.
For example, the rheology is a function of temperature, englacial water content, and damage from crevassing.
Likewise, the friction coefficient can be described in terms of the subglacial water pressure and a roughness factor for the underlying bedrock \citep{cuffey2010physics}.
The diagnostic and prognostic equations can then be supplemented with evolution equations for these fields, for example the heat equation for temperature, or a hydrology model for subglacial water pressure.

\begin{table}[h]
    \begin{tabular}{l|l}
        Symbol & Meaning \\
        \hline
        $h$ & thickness \\
        $b$ & bed elevation  \\
        $s$ & surface elevation  \\
        $d$ & water depth \\
        $u$ & velocity \\
        $\dot a_s$ & surface mass balance \\
        $\dot a_b$ & basal mass balance \\
        $\dot\varepsilon$ & strain rate \\
        $C$ & bed friction coefficient \\
        $A$ & rheology coefficient \\
        $\nu$ & unit outward normal \\
        $J$ & action functional
    \end{tabular}
    \caption{Mathematical symbols}
\end{table}

\begin{table}[h]
    \begin{tabular}{l|l}
        Symbol & Meaning \\
        \hline
        $n$ & Glen's flow law exponent \\
        $m$ & Weertman sliding law exponent \\
        $\rho_I$ & ice density \\
        $\rho_W$ & seawater density \\
        $g$ & gravitational acceleration
    \end{tabular}
    \caption{Physical constants}
\end{table}


\subsection{Prognostic model}

The \emph{prognostic model} or \emph{mass transport equation} (the two terms are synonymous) describes how the ice thickness changes in time.
The prognostic model is a conservative advection equation:
\begin{equation}
    \frac{\partial h}{\partial t} + \nabla\cdot hu = \dot a_s - \dot a_b,
\end{equation}
where $\dot a_s$ and $\dot a_b$ are the surface and basal mass balance.
Icepack represents the thickness using continuous, piecewise polynomial basis functions in each cell of the mesh.
We have not yet implemented a formulation that works with discontinuous basis functions, but this extension is completely feasible within our framework.

The surface elevation is calculated as
\begin{equation}
    s = \max\{b + h, (1 - \rho_I / \rho_W)h\},
\end{equation}
the first case corresponding to grounded ice and the second case corresponding to floating ice.
In the immediate vicinity of the grounding line, the assumption that the floating ice is in hydrostatic equilibrium with the ocean fails.
Most models assume hydrostasy and icepack does as well.
Elmer/Ice, on the other hand, solves a contact problem for the moving upper and lower ice surfaces and thus can accurately model non-hydrostatic ice shelves \citep{gagliardini2013capabilities}.
We have not implemented this feature but support for non-hydrostatic ice shelves may come in a future release of icepack.


\subsection{Diagnostic model}

There are four \emph{diagnostic} or \emph{momentum transport} models implemented in icepack.
For each of the diagnostic models, we use a formulation of the physics based on a \emph{minimization} or \emph{action principle} \citep{dukowicz2010consistent}.
Action principles are completely equivalent to the usual weak form of a partial differential equation but have certain numerical advantages, as described below.

The most complex and most physically accurate model that we implement is the \emph{first-order} or \emph{Blatter-Pattyn} approximation, a 3D system describing the horizontal velocity \citep{blatter1995velocity, pattyn2003new}.
(In icepack we refer to our implementation of this equation as a \emph{hybrid model} as it includes both shear and plug flow modes, described below.)
The only approximation that the first-order model makes is that the flow has a low aspect ratio -- the thickness of the glacier is much less than its horizontal extent.
This approximation may be questionable around, say, the main trunk of Jakobshavn Isbrae in Greenland, which flows through a very deep and narrow trough.
Even Jakobshavn has an aspect ratio on the order of 1/5 and almost all glacier flows have an aspect ratio less than 1/10 or even 1/20.

From the first-order model, two approximations are possible.
First, the \emph{shallow ice approximation} (SIA) comes from assuming that vertical shear is the dominant mode of ice flow.
The SIA model \citep{hutter1981effect} describes the interior of ice sheets well, where flow can be simply described as bed-parallel shear and surface and basal slopes are small.
This approximation breaks down near ice margins, where basal sliding can be a large fraction of the surface speed and where membrane stresses are substantial.
Second, one could assume that ice flow is purely by horizontal extension and that the surface and bed velocities are practically the same, which is called the \emph{shallow stream approximation} (SSA).
The SSA model describes the fast-flowing margins of the ice sheet best, encompassing grounded ice streams or floating ice shelves \citep{macayeal1989large}.

All the diagnostic models inherit a fundamental nonlinearity in the mechanics of ice flow.
For a Newtonian viscous fluid, the stress tensor $\tau$ and the strain rate tensor $\dot\varepsilon$ are linearly proportional to each other.
Glaciers, however, have a nonlinear constitutive relation:
\begin{equation}
    \dot\varepsilon = A|\tau|^{n - 1}\tau
\end{equation}
where $A$ is the \emph{fluidity} and $n$ is the \emph{Glen flow law exponent} \citep{cuffey2010physics}.
The most commonly-used value of the flow law exponent is $n \approx 3$ as determined from laboratory experiments.
Strictly speaking, the fluidity should be a tensor field, but is almost universally treated as a scalar.

All of the diagnostic models in icepack are described through \emph{variational} or \emph{action} principles \citep{dukowicz2010consistent}.
Rather than describe the velocity as the weak solution of a nonlinear PDE, an action principle instead states that the velocity minimizes a functional called the action.
The action consists of four terms:
\begin{align}
    & \text{action} = \iint\text{stress} \times \text{strain rate}\,dz\,dx - \int\text{basal friction} \times \text{sliding velocity}\,dx \nonumber \\
    & \quad - \iint\text{surface slope}\times\text{velocity}\,dz\,dx - \iint\text{ocean pressure}\times\text{velocity}\,dz\,d\gamma
    \label{action-functional}
\end{align}
where $dz\,dx$ denotes integration over the entire glacier, $dx$ denotes integration over the glacier footprint, and $dz\,d\gamma$ over the side wall boundary.
The action has units of power (energy/time) and can be related to the rate of decrease of the thermodynamic free energy.
Moreover, the energy lost to viscous and frictional heating can be calculated from the action and its Legendre transform.
The action principle can be viewed as a consequence of the Onsager reciprocity relations for systems near to equilibrium \citep{de2013non}.

Every diagnostic model in icepack is encapsulated in its own Python class.
The key responsibility of the model classes is to take in the input fields -- the ice thickness, velocity, etc. -- and return a symbolic description of the action functional in UFL.

The action principle is especially useful for designing a robust numerical solver.
For viscous flow problems near to steady-state, the action is \emph{convex} as a function of the ice velocity, i.e. its second derivative is positive-definite.
Convexity implies that the action functional has a unique minimizer and that, with an appropriate line search strategy, Newton's method will converge from any initial guess.
Minimizing a convex action functional is vastly more convenient numerically than solving a general nonlinear equation while having no additional computational cost.
Both formulations are mathematically equivalent.

\subsubsection{Shallow ice approximation}

The shallow ice model can be derived from the Blatter-Pattyn approximation by assuming ice flow is dominated by bed-parallel shear and that surface and basal slopes are small, i.e. $\partial u/\partial x \ll \partial u/\partial z$.
The result is a 2D system of equations for depth-averaged horizontal velocity.
The class in icepack that represents this physics model is called \texttt{ShallowIce}.
The terms in the action functional are:
\begin{align}
    \text{mass} & =\frac{1}{2}\int_\Omega u\cdot u\,dx\\\
    \text{gravity} & = \int_\Omega\frac{2A(\rho_I g)^n}{n+2} h^{n + 1}|\nabla s|^{n - 1}\nabla s\cdot u\,dx\\
    \text{penalty} & = \frac{1}{2}\int_\Omega \ell^2\nabla u\cdot \nabla u\,dx
\end{align}
and the action functional is
\begin{equation}
    J = \text{mass} + \text{gravity} + \text{penalty}.
\end{equation}
The default value of the length scale $\ell$ in the penalty term is defined as:
\begin{equation}
    \ell = 2\max\{\text{cell diameter}, 5h\}
\end{equation}
but users can adjust this to the value of their choice.
This penalty term smooths over numerical artifacts, especially near the ice margins and termini.
In these regions the shallow ice approximation is less applicable, so the error in solving a different set of equations is small compared to the inherent modeling error in using these equations in the first place.

We verified the correctness of our implementation by check that numerical results converge at the expected rate to the analytical Beuler profile for a circular symmetric ice sheet \citep{greve2009dynamics}.

The shallow ice approximation applies well in ice-sheet interiors.
For this reason, and because the equations are particularly simple to solve, this approximation has been a common choice for ice-sheet modeling \citep{cuffey2010physics, kirchner2016shallow}.
This approximation does not work well in areas of the ice sheet where the flow has substantial membrane stresses, like fast outlet glaciers.

\subsubsection{Shallow stream and shelf approximations}

The shallow stream model can be derived from the Blatter-Pattyn approximation by assuming nearly plug flow, i.e. $\partial u/\partial z \ll \partial u/\partial x$.
The momentum equations can, again, be vertically integrated to obtain a 2D system of equations.
The class in icepack that represents this physics model is called \texttt{IceStream}.
The terms in the action functional are:
\begin{align}
    \text{viscosity} & = \frac{n}{n + 1}\int_\Omega hA^{-\frac{1}{n}}|\dot\varepsilon(u)|^{\frac{1}{n} + 1}dx \\
    \text{friction} & = \frac{m}{m + 1}\int_\Omega C|u|^{\frac{1}{m} + 1}dx \\
    \text{gravity} & = -\int_\Omega\rho_I gh\nabla s\cdot u\,dx \\
    \text{terminus} & = \frac{1}{2}\int_\Gamma(\rho_I gh^2 - \rho_Wgd^2)u\cdot \nu\,d\gamma
\end{align}
and the action functional is
\begin{equation}
    J = \text{viscosity} + \text{friction} - \text{gravity} - \text{terminus}.
\end{equation}
When the ice is floating, there are two simplifications: the friction coefficient $C$ is 0, and the surface elevation $s$ can be written in terms of the thickness $h$ as
\begin{equation}
    s = (1 - \rho_I / \rho_W)h.
\end{equation}
Additionally, the terminal stress term of the action disappears by applying integration by parts to the gravity term.
Since the action functional for ice shelves has fewer terms than for grounded ice streams, we have defined a separate \texttt{IceShelf} model class.
The ice shelf and ice stream models share common components, i.e. the viscosity and side wall stress.

We verified the correctness of the ice shelf model by checking that numerical results converge at the expected rate to an analytical solution for the velocity with a linearly sloping thickness in a rectangular domain \citep{greve2009dynamics}.
When there is basal friction, the shallow stream equations do not have a simple analytical solution.
To get an exact solution, we chose the ice velocity and thickness a priori and used the computer algebra system SymPy \citep{sympy} to generate a friction coefficient that makes these fields an exact solution.
We then checked that numerical results converge to this manufactured solution at the expected rate.

Several studies have compared the shallow stream approximation to 3D models such as Blatter-Pattyn or full Stokes \citep{pattyn2013grounding}.
The most appreciable difference between lower- and higher-order models occurs near the glacier grounding line, where the full Stokes equations can represent bridging stresses \citep{van2013fundamentals}.
The lack of vertical strain rates in the ice viscosity in 2D models can also lead to different equilibrium grounding line positions under the same external forcing.
The advantage of the SSA is that that it can capture most of the overall flow features of an individual fast-flowing glacier but at much lower computational cost than the Stokes equations.


\subsubsection{Hybrid model} \label{sec:physics-hybrid-model}

The first-order model in icepack is described in the class \texttt{HybridModel}.
The shallow ice and shallow stream models follow directly from the variational principles described above.
The hybrid flow model, while also based on a variational principle, uses two more advanced mathematical techniques: terrain-following coordinates and spectral discretization in the vertical dimension.

Each of these techniques has appeared in the literature on glacier modeling before but rarely all in the same place for a 3D model.
\citet{langdon1978numerical} and \cite{bassis2010hamilton} used variational principles and vertical spectral methods, but these works considered only flowband models.
\citet{kleiner2014numerical} used terrain-following coordinates for 3D glacier flow modeling, but they discretized the problem with finite difference methods in every direction and did not take into account the variational formulation of the diagnostic model.
\citet{jouvet2015multilayer} used vertical semi-discretization of the variational problem, but this model used a finite difference discretization in the vertical direction.
The model used in \citet{brinkerhoff2015dynamics} is the closest to the one we present below.
This work used both terrain-following coordinates and a tensor product basis of Lagrange finite elements in the vertical and higher-degree polynomials in a single vertical layer.
For the vertical basis functions, they used a plug flow mode and one shear mode.

\textbf{Terrain-following coordinates}.
Rather than the usual Cartesian coordinate system, the hybrid flow model uses terrain-following coordinates.
The terrain-following vertical coordinate $\zeta$ is
\begin{equation}
    \zeta = \frac{z - b}{h}
\end{equation}
where $b$ is the ice base.
We can then think of the computational domain as the Cartesian product of a 2D footprint domain $\Omega$ and the unit interval $[0, 1]$.

Both the bed elevation and thickness depend on $x$ and $y$.
As a result, the formula for the horizontal gradient of a field in terrain-following coordinates includes an additional geometric correction factor.
Letting $\nabla_z$ and $\nabla_\zeta$ denote the horizontal gradient with respectively $z$ and $\zeta$ held constant, the chain rule gives us that
\begin{equation}
    \nabla_zq = \nabla_\zeta q + \frac{\partial q}{\partial\zeta}\nabla\zeta,
\end{equation}
where we can calculate the spatial gradient of $\zeta$ as
\begin{equation}
    \nabla\zeta = -h^{-1}\left\{(1 - \zeta)\nabla b + \zeta\nabla s\right\}.
\end{equation}
Likewise, the strain rate of a vector field can be expressed as
\begin{equation}
    \dot\varepsilon_z(u) = \dot\varepsilon_\zeta(u) + \frac{1}{2}\left(u \otimes\nabla\zeta + \nabla\zeta\otimes u\right)
    \label{eq:geometric-correction}
\end{equation}
where $\otimes$ is the tensor product of two vectors.

For the Stokes equations, this alternative coordinate system also helps avoid the problem of how to enforce the condition $u\cdot \nu = -\dot a_b$ at the ice base, where $\nu$ is the unit outward normal vector and $\dot a_b$ is the basal mass balance.
This boundary condition is difficult to impose exactly because the unit outward normal vector $\nu$ is defined on mesh faces while the velocity is defined at mesh vertices.
Elmer/Ice uses an ad-hoc procedure to define the unit normal vectors at mesh nodes \citep{gagliardini2013capabilities}.
This procedure is nearly always effective in practice.
But with a transformation to terrain-following coordinates, we can set the terrain-following vertical velocity $\omega$ to be $-\dot a_b/h$ at the ice base to impose this boundary condition with no additional intervention.

We also argue that answers expressed in terrain-following coordinates are more intuitive in some respects than in Cartesian coordinates.
At the bed of a grounded glacier, the vertical velocity in Cartesian coordinates is
\begin{equation}
    w = -\dot a_b + u\cdot\nabla b.
\end{equation}
Knowing that a model gives a vertical velocity at the base of a glacier of, say, 10 cm/year, the modeler needs to also know the bed slope and sliding velocity.
In other words, it is not immediately clear whether the vertical velocity is a result of basal mass balance or of geometry without additional information.
By contrast, the vertical velocity $\omega$ in terrain-following coordinates evaluated at $\zeta = 0$ is completely determined by basal mass balance and ice thickness.


\textbf{Spectral discretization}.
Terrain-following coordinates open up several choices for how to describe the vertical variation of the velocity field.
In Elmer/Ice, for example, the user can extend the finite element discretization into a number of vertical layers.
The number of vertical layers is a user-tuneable parameter, depending on the desired resolution along this axis \citep{gagliardini2013capabilities}.

The horizontal velocity for many realistic flows is very smooth as a function of depth and this suggests a different approach.
For example, under the plug flow approximation, the horizontal velocity is constant with depth.
Under the shallow ice approximation, the horizontal velocity varies with depth as $1 - (1 - \zeta)^{n + 1}$ where $n = 3$ is the Glen flow law exponent.
This extra information about our solution suggests a modal rather than a nodal discretization strategy.

Rather than divide the spatial domain into many vertical layers, we can instead use only one vertical layer and increase the polynomial degree in the vertical direction to obtain higher resolution.
This type of basis, in which different shape functions are used in different dimensions, is called a \emph{tensor product} element \citep{mcrae2016automated}.
Given a set of finite element basis functions $\{\phi_k(x, y)\}$ defined on the 2D domain $\Omega$ and a set of basis functions $\{\psi_l(\zeta)\}$ defined on the unit interval $[0, 1]$, the tensor product finite element basis $\{\Phi_{kl}\}$ on the extruded domain is defined as
\begin{equation}
    \Phi_{kl}(x, y, \zeta) = \phi_k(x, y)\psi_l(\zeta).
\end{equation}
For example, we can use piecewise linear or quadratic elements on triangles and use quintic or higher degree polynomials in the vertical.
Rather than use the usual Lagrange interpolating polynomial basis in the vertical dimension, we can instead use the Legendre polynomial basis.
The Legendre polynomials are mutually orthogonal and this choice makes the mass matrix block-diagonal.
Our main reason for choosing to build icepack using the Firedrake package was because it natively supports tensor product elements.

The combination of using extruded meshes and tensor product elements in the vertical direction can be thought of merely as a way to discretize a PDE that has special structure.
Alternatively, we can view discretization in the vertical as defining a family of models indexed by the number of vertical basis functions.
The order-$d$ model defines a coupled system of PDEs for $d$ vector fields.
Each vector field represents one mode of vertical variability, similar to the distinction between barotropic and baroclinic modes in atmospheric physics and oceanography.
The system is then discretized in the horizontal and solved numerically.
In any case, the code is the same regardless of how one views the underlying mathematics.

The user then has to decide how many vertical modes are sufficient.
Using only degree 0 is exactly equivalent to the shallow stream approximation and we use this fact as a ``smoke test'' for the hybrid model.
The degree 2 model is the most minimalistic model that still exhibits vertical shear and can satisfy the zero-stress boundary condition at the ice surface.
Going to higher degree gives a more accurate approximation at the expense of greater computational effort.

\citet{brinkerhoff2015dynamics} used a similar approach to the one described above, with a one vertical basis function for plug flow and one for shear flow.
The shear flow basis function was chosen to be exact assuming the SIA balance with cold ice, together with a heuristic approximation for polythermal ice.
By contrast, our approach allows for an arbitrary number of shear modes, although using a high-degree basis in the presence of a cold-temperate transition surface will introduce ringing artifacts.
In principle, one could confront this problem by using two layers -- one cold and one temperate -- instead of a single layer.
The extruded mesh capabilities in Firedrake provide the needed functionality for implementing such a model.
We have not yet implemented a multilayer model but this will be the subject of future work.

\subsection{Substituting model components} \label{sec:physics-substitution}

Many aspects of glacier physics are not completely understood.
For example, the most commonly used sliding law is the power law
\begin{equation}
    \tau_b = -C|u|^{\frac{1}{m} - 1}u
    \label{eq:weertman-sliding}
\end{equation}
for some exponent $m$.
Older research assumed $m = 3$ based on the theory of regelation \citep{weertman1957sliding}, which has since been referred to as the Weertman sliding law.
When $m = \infty$, the basal shear stress is independing of the sliding speed; this is referred to as perfect plasticity.
More recently, \citet{schoof2005effect} proposed an alternative form that acts like the Weertman law when $|u| \ll u_c$ and like the plastic law when $|u| \gg u_c$.
A simplified form of this model can be expressed as
\begin{equation}
    \tau_b = -C\left(\frac{|u|}{|u| + u_c}\right)^{\frac{1}{m}}\frac{u}{|u|}
\end{equation}
where $u_c$ is some critical speed.
This latter equation has been found to agree best with laboratory experiments on till \citep{zoet2020slip} and in reproducing observed velocity variations \citep{joughin2019regularized}.

The Weertman and plastic sliding laws possess the same functional form but differ only in the value of a single scalar parameter $m$.
The Schoof sliding law, on the other hand, has a totally different functional dependence on the velocity.
Several authors, including Schoof, have proposed that the basal shear stress is also a function of the effective pressure $N = \rho gh - p_w$ within the subglacial hydrological system \citep{budd1979empirical, schoof2005effect}.
Implementing these more sophisticated mathematical models would require adding an extra argument to the procedure for solving the diagnostic equation.

One of our goals with icepack is to facilitate experimentation with the model physics, even for novice users.
Of the programming languages that are commonly used for scientific computing, only Python and possibly Julia would appear to meet these needs.
To support use cases like implementing the Schoof sliding law, it must be possible not just to change the value of a single parameter but to completely alter the functional form of a given model component.
For uses cases like explicitly adding the dependence of basal shear stress on hydrology, it must also be possible to add entirely new fields to a given model component.
In programming terms, this amounts to changing the number of arguments to the function that calculates basal shear stress, which Python accomodates easily.
For a library developed in C or Fortran, the user would then also have to change the signature of the diagnostic solve function, which is undesirable.
In C++, one could avoid changing the signature of the diagnostic solve routine by (1) using variadic templates, (2) wrapping the inputs in a class, or (3) passing all arguments in a dictionary.
Using variadic templates or wrapping the inputs in a class would require users to know more about generic or object-oriented programming than a novice might.
Using a dictionary data structure to pass arguments is relatively easier but would be more idiomatic in Python than in C++.

In icepack, users can substitute any diagnostic model component for the parameterization of their choosing, including adding new fields.
From the user's perspective, substituting model physics components does not require any advanced language features beyond keyword arguments.
To understand how this is possible, we'll briefly describe the path that the input fields take through the program.
\begin{enumerate}
    \item The user passes all arguments to the diagnostic solve procedure by keyword.
    \item The diagnostic solve procedure creates a symbolic representation of the action functional by summing up several terms, like the viscosity, basal friction, etc.
        Calculating these terms is delegated to specialized procedures for each term.
        Each term procedure gets the entire collection of fields.
    \item The routine that calculates the terms of the action selects which fields it actually needs from the argument dictionary, and then creates the symbolic representation of that term.
    \item Finally, once the symbolic representation of the action functional has been created, all the responsibility passes to the nonlinear solver.
\end{enumerate}
To substitute different model components, the user intervenes at step 3.
Each model object -- shallow ice, shallow stream, etc. -- is initialized with a default set of routines to calculate the terms of that model's action functional.
The user can replace these default routines with one of their own choosing by passing the function of their choice when that model object is initialized.

In step 3, any fields that were unnecessary for calculating a given component are ignored.
For example, the gravitational driving power routine will pull out the velocity, thickness, and surface elevation.
While the routine will not use them, it also has access to other fields, for example the ice fluidity.

In adopting this approach, we are restricted to using keyword arguments instead of positional arguments.
We argue that employing only keyword arguments is a strength rather than a weakness because it enhances readability and comprehensibility for the particular use case of calling a physics solver.
The user only needs to know the argument names, which are chosen to agree with the English name most commonly used in the literature -- ``friction'', ``rheology'', ``velocity'', etc.
The order of the arguments is arbitrary and immaterial.
The preference for argument passing by name is specific to this use case, however, and is not universal.

\subsection{Heat transport} \label{sec:heat-transport}

We implemented the enthalpy transport model described in \citet{aschwanden2012enthalpy}.
The enthalpy describes the heat content of the material in a way that incorporates both temperature and latent heat stored in meltwater.
The temperature and meltwater fraction can be uniquely calculated at any point from the value of the enthalpy, so nothing is sacrificed in describing heat content one way or the other.
Using the enthalpy has the advantage of circumventing many of the difficulties associated with tracking the interface between cold and temperate ice.
Our implementation of the model uses all of the simplifying assumptions described in \citet{aschwanden2012enthalpy}, for example that horizontal diffusion is negligible and that heat capacity and conductivity are not temperature-dependent within each phase.

The fluidity factor $A$ in Glen's law is roughly a known function of both temperature and melt fraction and we have included this function in the package.
Having some description of heat transport is thus necessary for describing realistic glacier flows, but it is not sufficient by itself.
For one, while the dependence of fluidity on temperature is known fairly well from laboratory experiments, the dependence on melt fraction is known with much less certainty.
It is likely that users of this module will want to substitute in their own parameterization for melt fraction dependence.
Second, other processes such as damage, fabric development, and impurities can affect the fluidity as well.
For these reasons, the diagnostic model takes in the fluidity as an argument and users must calculate the fluidity field themselves, either using the default parameterization that icepack supplies or one of their own.
Our general design principle is that icepack will solve differential equations for the various prognostic fields but it's up to users to decide how these fields are coupled.

We have not implemented a model for the surface or englacial transport of meltwater, although it would be possible to include this process.
Instead, we rely on an external scheme to act as a sink of enthalpy for meltwater fraction values above some user-defined critical value.
A common choice for this critical value is 1\% \citep{aschwanden2012enthalpy}.

\subsection{Damage transport} \label{sec:damage-transport}

Other physical fields besides temperature can influence ice fluidity.
At large spatial scales (> 5 km), crevasse fields affect ice flow by reducing the lateral area over which stress can be transmitted.
Modeling individual fractures is not computationally feasible for large-scale simulations.
Instead, we have implemented the phenomenological model described in \citet{albrecht2014fracture}, which is based on the theory of continuum damage mechanics.
This model is defined in the class \texttt{DamageTransport}.
Prognostic damage models can be broken down into three parts: (1) evolve the damage field based on the membrane stress of the glacier, (2) advect the damage field with flow, and (3) feed the damage field back into the fluidity of the glacier.
Changes in fluidity in turn affect the membrane stress; the coupling between bulk damage and ice flow goes both ways.
The model from \citet{albrecht2014fracture} adds sources of damage where the membrane stress exceeds a critical value and sinks of damage when the principal strain rate is less than a critical value.
These kinds of simplified models ignore complications such as inertial or 3D effects.
They are intended more to capture some important features of the real physical system ecnomically than to describe it exactly.


\section{Data assimilation}

Icepack includes a set of routines for estimating the basal friction or rheology coefficients from observational data.
The class \texttt{InverseProblem} represents the specification of an inverse problem, which requires:
\begin{itemize}
    \item the model object and the method that solves the diagnostic equation,
    \item the objective and regularization functionals,
    \item the observed field and the name of the argument to the diagnostic solver,
    \item an initial guess for the field to be estimated and the name of the argument to the diagnostic solver, and
    \item extra data passed to the diagnostic solver such as boundary conditions.
\end{itemize}
This class currently assumes that the observed state is always the ice velocity.
In principle the same design would suffice for more complicated inverse problems and this is an area of active development.

The inverse problem class is flexible enough to account for users defining their own parameterization for the rheology or friction coefficient as described in \S\ref{sec:physics-substitution}.
This flexibility with respect to parameterization is not just convenient but essential for common data assimilation workflows.
Nearly all studies in the literature introduce a parameterization of the field to be estimated in terms of some auxiliary field in order to guarantee positivity \citep{macayeal1992basal, joughin2009basal}.
For example, one could define the friction coefficient $C$ in terms of an auxiliary variable $\beta$ as $C = \beta^2$ to guarantee positivity.
One could also just as easily use $C \propto \exp(\beta)$.
The data assimilation routines in icepack can work the same way for any parameterization of the physics because Firedrake provides a rich set of routines for symbolically calculating functional derivatives.
The inverse problem class only needs to know which functional needs to be differentiated with respect to which field.

The \texttt{InverseSolver} class is responsible for actually solving the inverse problem.
This class will be described further in section \S\ref{sec:numerics-inverse-solvers}.


\section{Numerics}

In the previous sections we described \emph{what} problems icepack can solve, i.e. various physics models and data assimilation problems.
In this section, we'll describe \emph{how} these problems are solved.
This separation between the two questions parallels the broader design of the software package.

The key classes that users interact with are flow models and solvers.
The role of the model classes is to describe what problem is being solved.
These classes describe the diagnostic model by taking in the input fields -- ice velocity, thickness, surface elevation, etc. -- and returning a symbolic representation of the action functional.
There are several model classes, one for each set of physics equations: \texttt{ShallowIce}, \texttt{IceShelf}, \texttt{IceStream}, \texttt{HybridModel}.
The model classes do not dictate how that problem should be solved numerically; this is the realm of a separate class called \texttt{FlowSolver}.
This flow solver class has methods for computing the solutions of the diagnostic and prognostic equations and works the same regardless of which model is being solved.
The diagnostic solve method amounts to invoking an external Newton solve procedure on the symbolic action functional that the model object creates.
The Newton solver itself is completely standard but the convergence criterion is not (see section \S\ref{sec:convergence-criteria}).
Finally, the flow solver has a method to update the ice thickness from the current value, the ice velocity, and the mass balance rates.

The Unified Form Language for specifying weak forms of PDEs contains all of the primitives necessary to express individual terms of the action functional.
These primitives consist of the basic vector calculus operators like the gradient of a field, tensor calculus operations like taking the dot product of two vectors or tensors, scalar functions like the square root or exponential, and symbolic integration over the mesh or its boundary.
For example, the strain rate for a given velocity field $u$ can be written as \texttt{sym(grad(u))}, where the function \texttt{grad} represents the symbolic gradient of a field and \texttt{sym} represents the symmetrization of a rank-2 tensor.


\subsection{Advective transport} \label{sec:prognostic-model}

The simplest explicit timestepping schemes are unstable with continuous Galerkin finite elements.
To solve the advection equation equation, other packages use the streamlined upwind Petrov-Galerkin (SUPG) method for the timestepping scheme \citep{brinkerhoff2013data, larour2012continental}.
The SUPG method with an explicit time discretization is conditionally stable \citep{donea2003finite}, but this scheme introduces a tuneable stabilization parameter.
In the interest of simplicity, we instead default to the implicit Euler scheme, which has no such parameters.
The implicit Euler scheme requires solving a non-symmetric linear system, but the computational cost comes with the advantage of unconditional stability.

Practicing glaciologists who have not studied numerical PDE may be unfamiliar with the Courant-Friedrichs-Lewy (CFL) condition.
The use of an unconditionally stable scheme guarantees that they will get an answer, rather than a runtime error, should they try to use a large timestep.
Users are still responsible for checking the accuracy of their results, for example by running at more than one resolution.
The extra computational cost of using an implicit time discretization for the prognostic equation is dwarfed by the cost of the diagnostic solve in any case.
Advanced users who are interested in maximizing performance can subclass the solver to implement a faster explicit scheme.

The implicit Euler scheme tends to diffuse out sharp discontinuities that may be present in the true solution \citep{donea2003finite}.
Since the ice thickness does not possess shockwaves or propagating discontinuities this error mode is tolerable.
The coupling of ice thickness to velocity makes the whole system more resemble a parabolic problem than a hyperbolic one, and under the shallow ice approximation the system is truly parabolic.

Other problems in glaciology have more of a hyperbolic character.
For example, the thresholding behavior of the source terms for the damage model (see \S\ref{sec:damage-transport}) can create sharp discontinuities.
The implicit Euler scheme would obscure this important feature.
For the damage solver, we have instead used a strong stability-preserving Runge-Kutta method in time and a discontinuous Galerkin basis in space to best capture these sharp interfaces \citep{shu1988efficient}.


\subsection{Convex optimization}

The action functional for each diagnostic model is convex, i.e. the second derivative is strictly positive-definite.
From a theoretical persepctive, convexity guarantees that the problem has a unique solution.
This property is also especially advantageous for implementing numerical solvers.
We use a damped Newton method to solve the diagnostic equations.
Starting from a guess $u_k$ for the velocity, the \emph{search direction} $v_k$ is the unique solution of the linear system
\begin{equation}
    d^2J(u_k)\cdot v_k = -dJ(u_k).
    \label{eq:newton-search-direction}
\end{equation}
The next approximation for the velocity minimizes $J$ along the search direction $v_k$ starting from $u_k$, i.e.
\begin{align}
    u_{k + 1} & = u_k + \alpha_k\cdot v_k, \nonumber\\
    \alpha_k & = \text{argmin}_\alpha \hspace{2pt}J(u_k + \alpha v_k).
\end{align}
For an initial guess sufficiently close to the exact solution, the undamped Newton method ($\alpha_k = 1$ at every step) converges quadratically.
The line search step ensures that the method can converge even from a poor initial guess, provided that the line search method satisfies the Armijo-Wolfe criteria \citep{nocedal2006numerical}.

For a convex problem, $d^2J$ is a symmetric and positive-definite matrix.
This has two advantages.
First, the search direction is always a descent direction for $J$, which is not always the case for non-convex problems.
Second, symmetry and positivity enable the use of specialized linear solvers, such as the Cholesky decomposition or the conjugate gradient algorithm, that are superior to their more general counterparts in many respects.

Other software packages that treat diagnostic models as nonlinear systems of equations tend to rely on ad-hoc procedures for initializing the numerical solution process.
For example, without a damping procedure in Newton's method, the iteration can prove unstable if initialized far away from the true solution.
Some packages combat this problem by using a few iterations of the more robust but slower Picard method first \citep{gagliardini2013capabilities}.
While this approach can be effective, it requires tuning the number of Picard iterations.
There is no guarantee that an adequate amount for one problem will work well on another.
This issue rarely appears on realistic input data, but when solving inverse problems, the intermediate guesses for the inferred field can be wildly unrealistic before converging.
A forward model solver that is not sufficiently robust can crash in these extreme scenarios.
By contrast, a damped Newton procedure using a line search that satisfies the Armijo-Wolfe criteria is guaranteed to converge on non-degenerate, if unrealistic, input data.


\subsection{Convergence metrics} \label{sec:convergence-criteria}

Several works in the literature have weighed the relative merits of different iterative methods for solving the nonlinear diagnostic equation \citep{perego2012parallel}.
Few have considered the problem of when to stop iterating.
The most common stopping criteria are when (1) the 2-norm of the residual is sufficiently small or (2) the relative change in the iterates is sufficiently small.
Each of these approaches has problems.
The residual norm depends on the discretization and does not weight all degrees of freedom proportionally, e.g. vertex and edge degrees of freedom in higher-order finite element methods.
The relative change criterion, on the other hand, can suggest convergence when in fact the method has stagnated.

We can devise a convergence criterion that works equally well, independent of the discretization and the quality of the initial guess, based on the \emph{Newton decrement} \citep{nocedal2006numerical}.
Since the second derivative operator $d^2J(u_k)$ is positive-definite, the Newton search direction $v_k$ computed from equation \eqref{eq:newton-search-direction} is a descent direction for $J$:
\begin{equation}
    dJ(u_k)\cdot v_k < 0.
\end{equation}
The absolute value of the quantity in the last equation is defined as the Newton decrement.
For $u_k$ sufficiently close to the true solution $u$, the Newton decrement roughly tells us how much we can expect the action to decrease:
\begin{equation}
    J(u_k) - J(u) \approx \frac{1}{2}|dJ(u_k)\cdot v_k|.
\end{equation}
We can then use the Newton decrement to decide when to stop the iteration, as described below.

As shown in equation \eqref{action-functional}, the action for most models has units of power and is the sum of dissipation due to viscosity, friction, gravitational driving, and ocean back-pressure at the terminus.
The viscous and frictional terms are convex, positive functions of the velocity.
The gravitational and terminus stress terms are linear in the velocity and can be of either sign.
If we define the \emph{scale functional}
\begin{equation}
    K(u) = \text{viscous dissipation} + \text{frictional dissipation}
\end{equation}
as only the positive parts of the action, then the convergence criterion
\begin{equation}
    |dJ(u_k)\cdot v_k| < \epsilon K(u_k)
\end{equation}
is independent of the discretization.
The intuition behind this criterion is that the iteration is halted when the expected decrease in the action functional is much smaller than the positive part of the action itself.

We have found empirically that, with this criterion and the Newton solver implementation in icepack, the iteration usually converges to machine precision in around 8 steps.
The iteration count can reach as high as 20 for exceptionally bad initial guesses for the velocity or with unphysical fluidity or friction values.
We also observe the expected doubling of the number of accurate digits in the value of the action once the velocity guesses are within the convergence basin of the true solution.
Other convergence criteria, such as using relative change in the velocity guesses, can terminate prematurely when the initial guess is very far outside the quadratic convergence basin.

The numerical solvers in icepack have been designed so that users who are not familiar with numerical optimization need not be confronted with a possibly bewildering array of algorithmic parameters.
Consequently, sensible defaults have been chosen for the Armijo and Wolfe criteria \citep{nocedal2006numerical}, and the tolerance for the line search is chosen based on that of the outer-level Newton iteration.
The Newton search direction is calculated using a direct factorization solver rather than, say, the conjugate gradient algorithm, as the use of another iterative method would introduce yet another algorithmic parameter.
Advanced users who are interested in performance optimization can change these algorithmic parameters by passing extra arguments to the solve procedure.


\subsection{Inverse solvers} \label{sec:numerics-inverse-solvers}

The \texttt{InverseProblem} class describes what problem is being solved, while the \texttt{InverseSolver} class is responsible for carrying out the numerical optimization.
There are three inverse solvers in icepack: a simple gradient descent solver, a quasi-Newton solver based on the BFGS approximation to the Hessian, and a Gauss-Newton solver.
All of these classes are based around the general idea of first computing a search direction and then performing a line search.
They differ in how the search direction is computed.

The gradient descent solver uses the search direction
\begin{equation}
    \phi_k = -M^{-1}dJ(\theta_k)
\end{equation}
where $M$ is the finite element mass matrix.
Gradient descent is a popular choice because the objective functional is always decreasing along this search direction.
However, the search direction can be poorly scaled to the physical dimensions of the problem at hand.
This method can be very expensive and brittle in the initial iterations and often takes many steps to converge.

The BFGS method uses the past $m$ iterations of the algorithm to compute a low-rank approximation to the inverse of the objective functional's Hessian matrix; see \citet{nocedal2006numerical} for a more in-depth discussion.
The BFGS method converges faster than gradient descent.
However, it suffers from many of the same brittleness issues in the initial iterations before it has built up enough history to approximate the Hessian inverse.

Finally, the Gauss-Newton solver defines an approximation to the ``first-order'' part of the objective functional Hessian.
Each iteration of Gauss-Newton is more expensive than that of BFGS or gradient descent because it requires the solution of a more complex linear system than just the mass matrix.
The Gauss-Newton method converges fastest by far in virtually every test case we have found, in some instances by up to factor of 50.

The derivative of the objective functional with respect to the unknown parameter is calculated using the adjoint method and the symbolic differentiation features of Firedrake.
The user does not need to provide any routines for the derivatives, only the symbolic form of the error metric and the regularization functional.
The model object is responsible for providing the symbolic form of the action functional.


\subsection{Hybrid model}

The hybrid flow model uses several features that are available in Firedrake to better exploit the special structure of the problem.
Implementing this model also required some mathematical sleight-of-hand related to the terminus boundary condition that has not appeared in the literature before.
Additionally, the hierarchical structure of spectral basis functions presents an opportunity for developing fast algorithms.
In all other respects the implementation of the hybrid flow model using convex optimization follows the techniques described above.

\subsubsection{Discretization}

In order to use terrain-following coordinates, the hybrid model assumes that the geometry of the domain is an extruded mesh, where a 2D footprint mesh is lifted into 3D.
Firedrake includes support for creating extruded meshes by calling the function \texttt{ExtrudedMesh} on the 2D footprint \citep{bercea2016structure, mcrae2016automated}.
The cells of an extruded mesh are triangular prisms instead of the more common tetrahedra used for general 3D meshes.
Not every 3D domain can be described by extruding a 2D domain, but the geometry of most glacier flow problems can.

The geometric correction factor in equation \eqref{eq:geometric-correction} for gradients in terrain-following coordinates can easily be represented in UFL.
By defining a wrapper around the UFL \texttt{grad} function, the code to define the action functional in terrain-following coordinates is only slightly more complex than in Cartesian coordinates.

For problems defined on extruded geometries, Firedrake includes support for tensor product elements, which includes using different bases in the horizontal and vertical directions \citep{mcrae2016automated}.
Tensor product elements are defined in Firedrake by passing the extra keyword arguments \texttt{vfamily}, \texttt{vdegree} to the constructor for a function space.
In our case, we used the usual continuous Galerkin basis in the horizontal and Gauss-Legendre elements in the vertical.
To select the Legendre polynomial basis, the user passes the keyword argument \texttt{vfamily=`Gauss-Legendre'} or \texttt{`GL'} for short to the constructor for the function space.

Extruded meshes and tensor product elements are available in Firedrake but not in FEniCS.
Other general-purpose finite element modeling packages that support tensor product elements include deal.II and nektar++ \citep{bangerth2007deal, cantwell2015nektar++}.
Like most other packages in this domain, deal.II and nektar++ are written in C++, whereas our goal for icepack was to have both the core and the user interface in Python.

\subsubsection{Ocean boundary condition}

Our approach for implementing a hybrid flow model works completely seamlessly but for one important detail.
The backpressure from ocean water at the calving front of a marine-terminating glacier is not a smooth function of depth.
The pressure is 0 above the water line and linearly increasing below it:
\begin{equation}
    \text{backpressure power} = \int_\Gamma\int_0^1 \rho_Wgh(\zeta_{\text{sl}} - \zeta)_+\,d\zeta\,d\gamma,
    \label{backpressure}
\end{equation}
where $\zeta_{\text{sl}}$ denotes the relative depth to the water line and the subscript $+$ denotes the positive part of a real number.
Were we to use the standard asssembly procedure in Firedrake to evaluate this integral, we would get an inaccurate result due to an insufficient number of integration points.
The resulting velocity solutions are then wildly inaccurate due to the mis-specification of the Neumann boundary condition.
A blunt solution to this problem would be to pass an extra argument to the Firedrake form compiler that specifies a much greater integration accuracy in the vertical for this term.
This fix reduces the errors in the velocities, but it does not eliminate them completely and it incurs a large computational cost.

\begin{figure}[h]
    \includegraphics[width=0.95\linewidth]{demos/legendre/pressure.png}
    \caption{The normalized ocean pressure ($p_w / \rho_Wg$) and Legendre polynomial approximations of several degrees (left), and the residuals of the approximation (right).
    For this particular example, the waterline is at $\zeta = 1/3$, which would be representative of a glacier grounded on a higher moraine.
    The moments of each of the residuals up to the approximation degree are all zero.}
    \label{fig:legendre}
\end{figure}

We instead implemented a routine that symbolically calculates the Legendre polynomial expansion of the function $(\zeta_{\text{sl}} - \zeta)_+$ with respect to the parameter $\zeta_{\text{sl}}$ using the package SymPy \citep{sympy}.
The symbolic variables for $\zeta$ and $\zeta_{\text{sl}}$ used in the SymPy representation of the polynomial expansion are then substituted for equivalent symbolic variables in Firedrake/UFL using the SymPy object's \texttt{subs} method.
The Legendre polynomial approximation to this function only converges linearly as the number of coefficients is increased, since the the function is continuous but not smooth, and the approximation exhibits noticeable ringing artifacts at high degree.
While the approximation itself is not very accurate, the calculated value of the integral in equation \eqref{backpressure} is exact because of the orthogonality property of Legendre polynomials.
Stated another way, the residuals in the approximation are large, but they integrate to 0 when multiplied by any Legendre polynomial up to the number of vertical modes.
An example of the pressure approximations using linear, quadratic, and cubic Legendre polynomials are shown in figure \ref{fig:legendre}.

The exact symbolic integration approach is both faster and more accurate than using a large number of quadrature points.
The same technique could be used to exactly calculate the ocean backpressure for any model, say the full Stokes equations, using terrain-following coordinates together with a Legendre polynomial expansion in the vertical.

\subsection{Performance}

Icepack largely inherits the performance capabilities of the Firedrake package, for which we refer to the benchmarks in \citet{rathgeber2016firedrake}.
Firedrake is in turn built on PETSc, which includes a rich suite of nonlinear solvers and preconditioners that have been demonstrated to scale up to hundreds of processors \citep{balay2019petsc}.
Firedrake also includes special features for defining sophisticated solvers and preconditioners that take advantage of problem-specific structure \citep{kirby2018solver}.
Icepack exposes these tools through the interface of the flow solver objects, so users can select any solver and preconditioner from PETSc.

We have mainly developed icepack for process-scale studies of individual glaciers or drainage basins.
For the demonstrations presented below, nearly all simulations run in a matter of minutes to hours on a single core.
Larger problems, such as continental-scale modeling, will require more sophisticated and possibly problem-specific approaches.
For example, a rudimentary preconditioner for the hybrid model system could use the degree-0 model as the coarse space in a multigrid-type approach.
These optimizations will be the subject of future work.

\section{Demonstrations}

The following demonstrations aim to show the capabilities of icepack on both synthetic and real problems.
The key features of icepack that these demonstrations aim to highlight are the variety of different physics models implemented and the flexibility of the components of these physics models.

\subsection{MISMIP+}

As a first test case for icepack, we ran the first experiment from the Marine Ice Sheet Model Intercomparison Project version 3 (MISMIP+).
The parameters and geometry for this experiment can be found in \citet{asay2016experimental}.
The MISMIP+ experiment has three phases.
First, the model must find a steady state marine ice sheet with a fixed accumulation rate and no submarine melting.
Next, submarine melting with a given depth-dependent parameterization is applied for 100 years.
The increased melt thins the ice ice shelf and initiates a retreat of the glacier grounding line.
Finally, submarine melting is turned off for at least 100 years, optionally longer.
The grounding line then readvances, but not as far as its original position.

The original intercomparison project specified that participants could use the Weertman sliding law (equation \eqref{eq:weertman-sliding}) as well as two other sliding laws that transition to a more plastic rheology at high sliding speeds.
The first alternative sliding law consists of Weertman sliding until the stress reaches a critical value, at which point the constitutive relation transitions to perfect plasticity.
The second alternative is the Schoof sliding law \citep{schoof2005effect}:
\begin{equation}
    \tau_b = -\frac{C|u|^{1/m}\cdot \tau_c}{(C^m|u| + \tau_c^m)^{1/m}}\frac{u}{|u|},
    \label{eq:schoof-sliding}
\end{equation}
where the critical stress $\tau_c$ is a certain specified fraction of the water pressure in the subglacial hydrological system.
Sliding laws in icepack are not expressed directly, but rather as the the derivative of an action functional.
To implement the Schoof sliding law we need to know the antiderivative of equation \eqref{eq:schoof-sliding}.
Using a computer algebra system, we found that the antiderivative of this expression in terms of hypergeometric functions.
The Unified Form Language has several transcendental functions (sine, cosine, exponential, etc.), but it does not currently support hypergeometric functions.
Instead, we implemented a sliding relation that exhibits the important features of the Schoof law, i.e. it behaves like an $m = 3$ power law at low sliding speeds and $m = \infty$ at high sliding speeds, but which is more tractable algebraically.
Knowing the critical stress $\tau_c$ and the friction coefficient $C$, which has units of stress $\times$ speed${}^{-1/m}$, we can define a \emph{critical speed} $u_c$ as
\begin{equation}
    u_c = C^{-m}\tau_c^m.
\end{equation}
The power dissipation density for the Schoof-type sliding law that we use is
\begin{equation}
    P = \tau_c\left\{\left(u_c^{\frac{1}{m} + 1} + |u|^{\frac{1}{m} + 1}\right)^{\frac{m}{m + 1}} - u_c\right\}
    \label{eq:power-dissipation-modified-schoof-sliding}
\end{equation}
Figure \ref{fig:sliding-laws} shows a comparison of the original Schoof sliding law and the sliding law that arises as the derivative of the functional in equation \eqref{eq:power-dissipation-modified-schoof-sliding}.
The two have the same asymptotic behavior when the speed is much smaller or much larger than the critical speed; they differ in a relatively small range around the critical speed.
The relative difference in basal shear stress between the two sliding laws is less than 10\% throughout the entire range.
Although figure \ref{fig:sliding-laws} shows a comparison of both sliding laws with the same value of the critical speed, by using different values, the agreement between our sliding law can be brought into much closer agreement with the Schoof law.

\begin{figure}[h]
    \includegraphics[width=0.95\linewidth]{demos/sliding/sliding-law.png}
    \caption{The Weertman, Schoof, and modified Schoof sliding law of equation \eqref{eq:power-dissipation-modified-schoof-sliding}.
    The critical speed and shear stress are $u_c = 250$ m/year and $\tau_c = 100$ kPa.}
    \label{fig:sliding-laws}
\end{figure}

To change the sliding law, users only need to pass one function or the other to the model object at initialization.
(See \emph{Code and data availability} below for the source code.)
Users do not need to implement a subclass that overrides some parent method.
This approach would be idiomatic in C++, but it requires knowledge of object-oriented programming that a non-expert might lack.

Figure \ref{fig:mismip-result} shows the thickness and ice speed after spinning up the MISMIP+ geometry and input data to steady state with the sliding law described above.
The spin-up used used degree-1 basis functions for both thickness and velocity.
To get a high-resolution estimate for the steady state, the spin-up started from a relatively coarse resolution to propagate out most of the transient signal.
Then the mesh was successively refined and spun up again to propagate out the remaining high-wavenumber transients.
This process was repeated three times.
The net result is that most of the spin-up is done at relatively little computational cost.
Figure \ref{fig:mismip-volumes} shows the total ice mass during the retreat and readvance phases of the experiment.
The initial response to turning on melting is very rapid but then becomes roughly linear with time.
When the high melt is turned off at the 100-year mark, the ice readvances, but the rate is much slower than the rate of decrease when melt was on.
This asymmetric response is typical and expected from ice physics; the rates broadly agree with the reference results computed with BISICLES in the original experimental specification \citep{asay2016experimental}.

We also ran the experimental setup to steady state using the hybrid flow model with vertical basis functions up to degree 2.
This is the most minimal set of vertical basis functions that can resolve plug flow and the stress boundary conditions at the ice base and surface.
The ratio of basal velocity to surface velocity is shown in figure \ref{fig:mismip-sliding-ratio}.
The areas with the most significant vertical deformation are at the inflow boundary and where the troughs at the side walls are steepest.
Otherwise, the sliding ratio is above 0.8 throughout almost the entire domain.

\begin{figure}[h]
    \includegraphics[width=0.95\linewidth]{demos/mismip/steady-state-result.png}
    \caption{The steady state thickness and velocity of the MISMIP+ experimental setup with the modified Schoof sliding law of equation \eqref{eq:power-dissipation-modified-schoof-sliding}.}
    \label{fig:mismip-result}
\end{figure}

\begin{figure}[h]
    \includegraphics[width=0.5\linewidth]{demos/mismip/volumes.png}
    \caption{Total ice mass history during retreat (first 100 years) and readvance (second 100 years) of MISMIP+ experiment.}
    \label{fig:mismip-volumes}
\end{figure}

\begin{figure}[h]
    \includegraphics[width=0.95\linewidth]{demos/mismip/sliding-ratio.png}
    \caption{Ratio of basal velocity to surface velocity in steady state of MISMIP+ scenario computed with hybrid model.}
    \label{fig:mismip-sliding-ratio}
\end{figure}

Using variational principles to express all constitutive laws is less flexible than specifying the sliding law directly and this is a distinct disadvantage.
We were nonetheless able to implement a sliding law that exhibits the important characteristics of the Schoof sliding law.
\citet{asay2016experimental} also suggest using a sliding law that transitions sharply to exact plasticity above the critical speed.
Expressing this sliding law in UFL requires a conditional in the velocity and is thus no longer differentiable, causing the forward solver to crash.
The numerical advantages of using variational principles are so great, however, that we view this tradeoff as acceptable.
Moreover, there are few physical reasons to prefer one form over another, as long as the asymptotic behavior is correct above and below the critical speed.


\subsection{Synthetic ice sheet}

\begin{figure}[h]
    \includegraphics[width=0.95\linewidth]{demos/ice-sheet/ice-sheet.png}
    \caption{Synthetic ice sheet simulation. (a) The bed elevation profile consists of a ring of mountains with valleys of varying depth surrounding a flat plateau where the ice sheet is initialized.
    The black square outlines the domain of subplot (b) which shows the velocity of the ice as it passes through the southernmost mountain pass, with contours of the bed elevation shown in greyscale.}
    \label{fig:ice-sheet}
\end{figure}

As a demonstration of the shallow ice approximation model, we ran an experiment inspired by \cite{kessler2008fjord}.
In this work, the authors coupled an ice flow model to simple models for bed-erosion, calving, and glacial isostatic adjustment (GIA) to investigate the formation of deep fjords that are characteristic of coastlines on Baffin Island, Greenland, and British Columbia, among others.
We emulated their domain and bedrock geometry -- a plateau surrounded by a ridge punctuated by four valleys -- without also simulating erosion or GIA.

Figure \ref{fig:ice-sheet} shows the results of this computational experiment.
We initialized the experiment with a simple but unrealistic ice thickness.
We then evolved the ice sheet without climate forcing for 500 years.
The ice sheet relaxes very rapidly in the first 200 years, but after this changes are much slower as ice must be funneled through one of the four narrow valleys.
The resulting velocity pattern is similar to that of \citet{kessler2008fjord}, with the highest velocities where ice flows out from the valleys.
In and upstream of the valleys, the surface is drawn down due to the elevated export of ice.

The diagnostic model used in this demonstration is computationally cheap enough that it can be used to simulate ice sheets over several millenia in a matter of minutes on a desktop.
A key feature of icepack is the ability to choose between many different diagnostic models.
Users are free to decide what model works best for the spatial and temporal scales of their problem, how accurately they need to solve this problem to produce useful results, and what computational resources they are willing to devote.


\subsection{Synthetic ice shelf}

\begin{figure}[h]
    \includegraphics[width=0.95\linewidth]{demos/gibbous/damage.png}
    \caption{Results of ice shelf damage simulation: (a) ice velocity without damage, (b) change in ice speed with and without damage, (c) thickness, and (d) damage field.}
    \label{fig:damage}
\end{figure}

We simulated the evolution of a synthetic ice shelf towards steady state and coupled it with the damage transport model described in \S\ref{sec:damage-transport}.
The geometry of the ice shelf consists of the intersection of two circles, one with a larger radius and offset center, designed to roughly mimic the shape and size of real ice shelves.
The radius of the whole shelf is 200km.
Four ice streams flow into the shelf with varying speeds and a prescribed inflow thickness.
In the first phase of the experiment, the ice shelf is propagated to steady state without damage for 200 years.
After this time interval, the flux imbalance is less than 1\% of the influx.
In the second phase, damage transport is turned on and coupled to the ice velocity.
An interesting feature to observe in the approximate equilibrium damage field is that the highest values occur between the streams and not within them.
Additionally, the spacing between the streams changes as a result of adding damage.
The final ice thickness, velocity, strain rate, and damage are shown in figure \ref{fig:damage}.


\subsection{Larsen C Ice Shelf}

To demonstrate the inverse solver, we will estimate the rheology of the Larsen C Ice Shelf in the Antarctic Peninsula from observational data.
Several recent studies have focused on Larsen C because it may be unstable in the warmer climate of the coming decades.
From January to March of 2002, the neighboring Larsen B Ice Shelf disintegrated due to surface melt pond-induced fracture \citep{banwell2013breakup}.
This mechanism might also lead to the breakup of Larsen C in the future.
One of the key factors affecting the stability of ice shelves is the presence of marine ice -- seawater that freezes onto the base of an ice shelf -- in the suture zones where two flow units meet \citep{kulessa2014marine}.
Marine ice is warmer than meteoric ice and usually includes brine pockets, which are discernible in radio echo sounding measurements as the absence of reflection from the ice shelf base \citep{holland2009marine}.
This warmer and impurity-laden ice is more ductile and thus should be less prone to fracture than cold and brittle meteoric ice.
Ocean models predict that marine ice forms under Larsen C as well \citep{holland2009marine}.
By estimating the material rigidity of an ice shelf, we can constrain where marine ice may be forming.


\subsubsection{Data}

We used the InSAR phase-based ice velocity map from \citet{mouginot2019continent}.
This dataset has nominal errors over the Larsen Ice Shelf on the order 1 -- 7 meters per year.
We used the recently-released BedMachine map of the thickness of Antarctica, which takes advantage of newly available remote sensing data \citep{morlighem2019deep}.

Existing work on glaciological inverse problems uses the mismatch between the computed and observed velocity fields as part of the objective functional.
The effect of thickness errors is studied largely through a posteriori validation \citep{joughin2004basal, larour2005rheology}.
Errors in thickness or surface slope can be large enough that it might be impossible to fit the velocity measurements to the degree that statistical theory predicts \citep{macayeal1995basal}.
The velocity measurements themselves might have significant outliers, in which case using the usual weighted sum of squared misfits as an error metric will give poor results.
For these reasons, some studies have explored alternative objective functionals \citep{morlighem2010spatial}.
We have opted to use the regularized $L^1$-type error metric
\begin{equation}
    E(u - u^o) = \int_\Omega\left(\sqrt{\frac{|u - u^o|^2}{\sigma^2} + \gamma^2} - \gamma\right)dx.
    \label{eq:l1-error-metric}
\end{equation}
This error metric approaches the usual weighted sum of squared errors as $\gamma \to \infty$, and approaches the sparsity-promoting $L^1$ error metric as $\gamma \to 0$.
For finite, positive values of $\gamma$ this error metric is robust to non-normality or a small fraction of outliers \citep{barron2019general}.


\subsubsection{Parameterization}

The rheology parameter of an ice shelf is strictly positive.
The optimization algorithm, however, can explore unphysical regions of parameter space without some a priori constraints.
Rather than try to solve an inequality-constrained problem, most studies in glaciology instead re-parameterize the problem in terms of some auxiliary field in such a way that the rheology is manifestly positive.
In this case we use the parameterization
\begin{equation}
    A = A_0e^{\theta}
\end{equation}
and estimate $\theta$.
The inverse solver calculates the derivative of the objective functional symbolically and is thus agnostic to the particular parameterization.
The only information that the user needs to pass is the name of the arguments to the forward model representing the parameter and the observed field so that these can be passed by keyword.


\subsubsection{Results}

\begin{figure}[h]
    \includegraphics[width=0.95\linewidth]{demos/larsen/parameter.png}
    \caption{Left: Inferred fluidity parameter $\theta$.
    Lower values indicate more deformable ice.
    Right: stream plot of ice velocity computed from this fluidity parameter.
    Background image is the MODIS Mosaic of Antarctica \citep{scambos2007modis, haran2014modis}, courtesy of NASA NSIDC DAAC.}
    \label{fig:larsen}
\end{figure}

The inferred parameter field $\theta$ is shown in figure \ref{fig:larsen}.
The algorithm detects areas of much lower ice rigidity around highly damaged ice.
This feature is especially apparent around the large rift emanating from the Gipps Ice Rise, as well as the crevassed areas upstream.
We find other areas of low rigidity in the suture zones where two flow units converge and where marine ice tends to form, releasing heat to the ice shelf, exactly as observed in \citet{holland2009marine}.
Finally, the inferred rheology field reproduces features that have already been found in previous studies of Larsen C Ice Shelf \citep{khazendar2011acceleration}.

The final value of the model-data misfit from equation \eqref{eq:l1-error-metric} matches the value we would expect if the velocity errors actually came from this probability distribution.
In a separate computational experiment, we used the older bedmap2 ice thickness \citep{fretwell2013bedmap2}.
We were unable to achieve the same model-data misfit using bedmap2 at the same grid resolution with any regularization parameter.


\section{Usability}

One of the main goals for icepack is to create a tool that is accessible to researchers who might not be experts in scientific computing.
Previous work on numerical modeling of glacier flow has focused largely on the technical details of the models themselves -- does a given solver converge with the accuracy expected from finite element theory; does it scale to large numbers of processors; can models accurately predict grounding line retreat, etc.
The person learning to use an unfamiliar software tool is largely absent from the discussion.
For graduate students or other researchers who are not experts in computational physics, the difficulty of learning to use a particular software package may be more of a rate-limiting factor than the speed or efficiency of that software.

The field of \emph{human-computer interaction} (HCI) asks how we can design software that is easier to learn and use effectively.
In the following, we will describe some of the design choices in icepack and how they relate to what HCI researchers call the \emph{cognitive dimensions of notations}.
\citet{green1996usability} introduced this concept to assess the usability of visual programming languages, but the criteria they laid out in their study have been used to analyze software systems across many disciplines.

\textbf{Consistency: After a user learns part of the software, can they guess the remaining parts?}
Each of the model objects in icepack is a class with a method ending in \texttt{solve} that takes in keyword arguments for the various input fields and options for things like boundary conditions.
Users already familiar with, say, the \texttt{IceStream} class can then use the \texttt{HeatTransport} class under the assumption that the input fields -- the current temperature $T$, ice velocity $u$, and basal melt rate $m$ -- are passed as keyword arguments with the same name as the fields themselves.
Consistency obviates the need for repeatedly consulting the documentation or examples once users are already familiar with the software.

\textbf{Progressive evaluation: How easily can users get feedback during their use of the software?}
Progressive evaluation is the main advantage of having a user interface in an interpreted programming language such as Python as opposed to a compiled language where programs can only run in batch mode.
In the early stages of the development process, any non-trivial simulation of a physical system exists as a prototype which may be non-functional or even broken.
The ability to manually step through a simulation and examine the entire state in an interpreter is critical to finding errors faster.
Icepack was designed to give fine-grained control over how simulations work in order to support this mode of debugging.
The API provides routines for solving the diagnostic and prognostic equations.
It is the user's responsibility for making the repeated calls to these routines, either in a loop or by manually iterating through one step at a time.
With this responsibility comes the freedom to add in arbitrary code.
This capability might be used to add sanity checks, such as printing minimum and maximum thickness or velocity values.
It can also be used to get feedback on how long the simulation will take, or to save results for later visualization.
Other packages support a more coarse-grained view where the user only specifies the start and end time of a simulation and has more limited options for inspecting the state of the running program.

Icepack makes extensive use of Jupyter notebooks as a form of executable documentation.
Jupyter notebooks are a document format that includes code and explanatory text with typeset mathematics that runs interactively in a web browser \citep{kluyver2016jupyter}.
The contents of a notebook can executed incrementally much like running code in the Python interpreter.
Most importantly, Jupyter notebooks can render and display visualizations on the fly.
This enables a workflow where plotting all intermediate results serves for sanity-checking an experimental simulation.

\textbf{Abstraction gradient: What are the levels of abstraction exposed by the library?
Can irrelevant details be hidden?}
The API for icepack has been designed so that the users only need to decide what problem to solve and not how to solve it.
Where a choice does concern more the ``how'' than the ``what'', we use a sensible default that biases for correctness rather than speed.
For example, the Newton solver uses a direct factorization method to solve the linear system for the search direction because factorization requires no tuning whereas iterative methods do.
A user interested in achieving greater runtime performance can pass additional keyword arguments instead specifying, say, the preconditioned conjugate gradient method.
This choice is of interest mostly to advanced users so we keep the linear solver algorithm as a default argument.
In so doing, we avoid confronting novice users with options that they might not understand.

Advanced users who do wish to tune solver performance for large simulations will need some way to make choices about algorithms.
For example, one might choose the GMRES iterative solver together with an incomplete LU preconditioner to solve linear systems.
The solver classes, as opposed to the model classes, provide the interface for making these choices.
While alternative solvers might offer faster runtime performance than direct factorization, they also requires making additional choices -- how often to restart GMRES?
How much fill-in to allow in the incomplete factorization?
Many glaciologists do not have the background in numerical linear algebra to know that adjusting these parameters could make the difference between solver convergence or breakdown.
As another example, users might want to select between different discretization strategies for the Stokes equations.
The discretization of the Stokes equations has to be chosen carefully in order to satisfy the Ladyzhenskaya-Babu\v{s}ka-Brezzi (LBB) conditions \citep{boffi2013mixed}.
When using Galerkin least-squares stabilization of the weak form, the user has to pick a value of the stabilization parameter.
Determining exactly what value of this parameter is necessary to guarantee stability is a subtle problem, even more so for the kinds of highly anisotropic meshes that are commonly encountered in 3D glacier flow modeling.
If the solver fails to converge, it might not be obvious even to an expert whether the problem lies with the stabilization or the aforementioned parameters of the linear solver.


\conclusions  %% \conclusions[modified heading if necessary]

We have presented a new software package called \emph{icepack} for modeling the flow of glaciers and ice sheets.
This package advances the state of the art in this field by providing a platform for easily experimenting with the model physics.
In this paper, we have presented three demonstrations of this feature:
\begin{enumerate}
    \item coupling a model of ice shelf flow to a phenomenological model of damage,
    \item changing the sliding law in a simulation of a marine-terminating glacier, and
    \item inferring the fluidity of a floating ice shelf in a way that guarantees positivity.
\end{enumerate}
The phyics of how glaciers interact with their environment are not completely understood.
Consequently, the ability to change components of the model is an essential feature for any tool aimed at researchers in this field.

We have also paid special attention to how we can design this software package to be most usable for its intended audience.
Usability is not often discussed in the literature on computational science.
We believe that this is because of two difficulties: (1) quantifying the degree to which usability is a rate-limiting factor and (2) concretely assessing what features make software tools more or less usable.
By contrast, measuring computational performance is much more feasible although still fraught with difficulties of its own.
(This is not to say that performance optimization is easy by any means.)
In working with several graduate students and postdoctoral researchers in glaciology, we have observed that usability is a substantial bottleneck for researchers at these career stages.
For this reason, we have chosen to focus explicitly on usability by drawing on the research literature in HCI.
We argue that the same design features that enhance usability for relative novices to the subject area will also enhance the productivity of expert users.

This paper has presented several physics models currently implemented in icepack along with demonstrations.
Future developments will include:
\begin{enumerate}
    \item an implementation of full Stokes flow,
    \item dynamic glacier termini to enable calving models, and
    \item adaptivity through moving mesh methods.
\end{enumerate}
New features will be guided by feedback from icepack users and from the glaciological community at large.
By providing implementations of several glacier flow models from less to more complex and by enabling users to experiment with the physics, this tool both lowers the barrier to entry for novices to glacier flow modeling and provides a pathway for these users to progress towards ever more sophisticated and advanced simulations.


%% The following commands are for the statements about the availability of data sets and/or software code corresponding to the manuscript.
%% It is strongly recommended to make use of these sections in case data sets and/or software code have been part of your research the article is based on.


\codedataavailability{
All code used in this repository is free and open source and all data sets used in the demonstrations are publicly available.
The icepack source repository is at https://github.com/icepack/icepack.
Icepack is released under the GPLv3 license.
The git commit hash of the version of the code used for this publication is b78b0ee571590cf5af242101c159154c6872b7b9, see also the Zenodo release at https://doi.org/10.5281/zenodo.1205640.
The icepack documentation and user manual is hosted at https://icepack.github.io.
The source code for the demonstrations used in this paper is hosted at https://github.com/icepack/icepack-paper, commit hash 4db3dc281e83b4304751ded3b77f166942b0efc6.
The demonstrations used glacier outlines that were hand-digitized from satellite imagery in order to generate the model domains.
These outlines are kept in version control and hosted at https://github.com/icepack/glacier-meshes and the git commit hash for the version used in this publication is c98a8b7536b1891611566257d944e5ea024f2cdf.
Additional observational data are hosted at the US National Snow and Ice Data Center (https://www.nsidc.org).
}

\appendix
% \section{}    %% Appendix A

% \subsection{}     %% Appendix A1, A2, etc.


\noappendix       %% use this to mark the end of the appendix section. Otherwise the figures might be numbered incorrectly (e.g. 10 instead of 1).

%% Regarding figures and tables in appendices, the following two options are possible depending on your general handling of figures and tables in the manuscript environment:

%% Option 1: If you sorted all figures and tables into the sections of the text, please also sort the appendix figures and appendix tables into the respective appendix sections.
%% They will be correctly named automatically.

%% Option 2: If you put all figures after the reference list, please insert appendix tables and figures after the normal tables and figures.
%% To rename them correctly to A1, A2, etc., please add the following commands in front of them:

\appendixfigures  %% needs to be added in front of appendix figures

\appendixtables   %% needs to be added in front of appendix tables

%% Please add \clearpage between each table and/or figure. Further guidelines on figures and tables can be found below.



\authorcontribution{
DS designed and implemented icepack with contributions from JB and AH.
IJ tested the model and assisted with design and debugging.
All authors contributed to the demonstration codes and to writing this paper.
} %% this section is mandatory

\competinginterests{
All authors declare that they have no competing interests.
} %% this section is mandatory even if you declare that no competing interests are present

\begin{acknowledgements}
We would like to thank the Firedrake development team for their support and for many helpful discussions.
\end{acknowledgements}

\textit{Financial support}.
DS was supported by US National Science Foundation grant \#1835321 and National Aeronautics and Space Administration grant \#80NSSC20K0954.
JB was supported by NSF grant \#1256082.
AH was supported by NASA grant \#80NSSC20K1627.
IJ was supported by NSF grants \#1835321 and \#1643285.


%% REFERENCES

%% The reference list is compiled as follows:

%\begin{thebibliography}{}

%\bibitem[AUTHOR(YEAR)]{LABEL1}
%REFERENCE 1

%\bibitem[AUTHOR(YEAR)]{LABEL2}
%REFERENCE 2

%\end{thebibliography}

%% Since the Copernicus LaTeX package includes the BibTeX style file copernicus.bst,
%% authors experienced with BibTeX only have to include the following two lines:
%%
\bibliographystyle{copernicus}
\bibliography{icepack.bib}
%%
%% URLs and DOIs can be entered in your BibTeX file as:
%%
%% URL = {http://www.xyz.org/~jones/idx_g.htm}
%% DOI = {10.5194/xyz}


%% LITERATURE CITATIONS
%%
%% command                        & example result
%% \citet{jones90}|               & Jones et al. (1990)
%% \citep{jones90}|               & (Jones et al., 1990)
%% \citep{jones90,jones93}|       & (Jones et al., 1990, 1993)
%% \citep[p.~32]{jones90}|        & (Jones et al., 1990, p.~32)
%% \citep[e.g.,][]{jones90}|      & (e.g., Jones et al., 1990)
%% \citep[e.g.,][p.~32]{jones90}| & (e.g., Jones et al., 1990, p.~32)
%% \citeauthor{jones90}|          & Jones et al.
%% \citeyear{jones90}|            & 1990



%% FIGURES

%% When figures and tables are placed at the end of the MS (article in one-column style), please add \clearpage
%% between bibliography and first table and/or figure as well as between each table and/or figure.

% The figure files should be labelled correctly with Arabic numerals (e.g. fig01.jpg, fig02.png).


%% ONE-COLUMN FIGURES

%%f
%\begin{figure}[t]
%\includegraphics[width=8.3cm]{FILE NAME}
%\caption{TEXT}
%\end{figure}
%
%%% TWO-COLUMN FIGURES
%
%%f
%\begin{figure*}[t]
%\includegraphics[width=12cm]{FILE NAME}
%\caption{TEXT}
%\end{figure*}
%
%
%%% TABLES
%%%
%%% The different columns must be seperated with a & command and should
%%% end with \\ to identify the column brake.
%
%%% ONE-COLUMN TABLE
%
%%t
%\begin{table}[t]
%\caption{TEXT}
%\begin{tabular}{column = lcr}
%\tophline
%
%\middlehline
%
%\bottomhline
%\end{tabular}
%\belowtable{} % Table Footnotes
%\end{table}
%
%%% TWO-COLUMN TABLE
%
%%t
%\begin{table*}[t]
%\caption{TEXT}
%\begin{tabular}{column = lcr}
%\tophline
%
%\middlehline
%
%\bottomhline
%\end{tabular}
%\belowtable{} % Table Footnotes
%\end{table*}
%
%%% LANDSCAPE TABLE
%
%%t
%\begin{sidewaystable*}[t]
%\caption{TEXT}
%\begin{tabular}{column = lcr}
%\tophline
%
%\middlehline
%
%\bottomhline
%\end{tabular}
%\belowtable{} % Table Footnotes
%\end{sidewaystable*}
%
%
%%% MATHEMATICAL EXPRESSIONS
%
%%% All papers typeset by Copernicus Publications follow the math typesetting regulations
%%% given by the IUPAC Green Book (IUPAC: Quantities, Units and Symbols in Physical Chemistry,
%%% 2nd Edn., Blackwell Science, available at: http://old.iupac.org/publications/books/gbook/green_book_2ed.pdf, 1993).
%%%
%%% Physical quantities/variables are typeset in italic font (t for time, T for Temperature)
%%% Indices which are not defined are typeset in italic font (x, y, z, a, b, c)
%%% Items/objects which are defined are typeset in roman font (Car A, Car B)
%%% Descriptions/specifications which are defined by itself are typeset in roman font (abs, rel, ref, tot, net, ice)
%%% Abbreviations from 2 letters are typeset in roman font (RH, LAI)
%%% Vectors are identified in bold italic font using \vec{x}
%%% Matrices are identified in bold roman font
%%% Multiplication signs are typeset using the LaTeX commands \times (for vector products, grids, and exponential notations) or \cdot
%%% The character * should not be applied as mutliplication sign
%
%
%%% EQUATIONS
%
%%% Single-row equation
%
%\begin{equation}
%
%\end{equation}
%
%%% Multiline equation
%
%\begin{align}
%& 3 + 5 = 8\\
%& 3 + 5 = 8\\
%& 3 + 5 = 8
%\end{align}
%
%
%%% MATRICES
%
%\begin{matrix}
%x & y & z\\
%x & y & z\\
%x & y & z\\
%\end{matrix}
%
%
%%% ALGORITHM
%
%\begin{algorithm}
%\caption{...}
%\label{a1}
%\begin{algorithmic}
%...
%\end{algorithmic}
%\end{algorithm}
%
%
%%% CHEMICAL FORMULAS AND REACTIONS
%
%%% For formulas embedded in the text, please use \chem{}
%
%%% The reaction environment creates labels including the letter R, i.e. (R1), (R2), etc.
%
%\begin{reaction}
%%% \rightarrow should be used for normal (one-way) chemical reactions
%%% \rightleftharpoons should be used for equilibria
%%% \leftrightarrow should be used for resonance structures
%\end{reaction}
%
%
%%% PHYSICAL UNITS
%%%
%%% Please use \unit{} and apply the exponential notation


\end{document}
